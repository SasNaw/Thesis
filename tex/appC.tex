\chapter{Tessellation Service Documentation}
\label{secC}
The TS is openly available at GitHub and comes with instructions for usage and installation. The Repository can be found at:

\url{https://github.com/SasNaw/TessellationService}

\section{Main}

\subsubsection{run()}

\begin{lstlisting}[frame=single,language=python]
def run(input):
	for element in input:
		# input is folder:
		if(os.path.isdir(element)):
			files_from_dir(element)
		# input is file:
		elif(os.path.isfile(element)):
			regions_from_file(element)
\end{lstlisting}


\subsubsection{files{\textunderscore}from{\textunderscore}dir(element)}

\begin{lstlisting}[frame=single,language=python]
def files_from_dir(dir):
	if not dir.endswith('/'):
		dir = dir + '/'
	contents = os.listdir(dir)
	for content in contents:
		if os.path.isdir(dir + content):
			if not content.endswith('_files'):
				files_from_dir(dir + content)
		else:
			regions_from_file(dir + content)
\end{lstlisting}


\subsubsection{regions{\textunderscore}from{\textunderscore}file(element)}

\begin{lstlisting}[frame=single,language=python]
def regions_from_file(file):
	if file.endswith('.dzi'):
		dzi(file)
	else:
		if(is_suppoted(file)):
			wsi(file)
\end{lstlisting}


%Once \texttt{save{\textunderscore}image(image, region, slide{\textunderscore}name, *tiles)} is called, the destination for the output is set, either to a location provided via -o or in the same directory as the TS. If the provided directory should not exist, it will be created. To keep correspondence to the label of the ROI, The provided image is saved in a directory with the name of the (see line 2 - 7).  

% # L = R * 299/1000 + G * 587/1000 + B * 114/1000

%and the annotated regions extracted from the corresponding JSON file. After the arguments and parameters are parsed, the script's \texttt{run()} method is called, which starts the extraction process for all input elements:

%As stated in section \ref{sec5_method}, each input element can be a file or a dictionary. Therefore, the individual entries must be examined. If the current element is a WSI or DZI, the extraction begins right away (see line 8). If it is a directory, \texttt{files{\textunderscore}from{\textunderscore}dir(dir)} will be called (see line 5) and search for valid files, including all contained subdirectories:

%After the contents of the directory are received (see line 4), they are evaluated further (see line 5 - 10). If the input directory contains subdirectories, \texttt{files{\textunderscore}from{\textunderscore}dir(dir)} is called recursively for each one of them, until the end of each directory tree is reached (see line6 - 8 in \texttt{files{\textunderscore}from{\textunderscore}dir(dir)}). Otherwise, the extraction process is started for the detected file (see line 9 - 10 in \texttt{files{\textunderscore}from{\textunderscore}dir(dir)}). Directories ending in \emph{"{\textunderscore}files"} are excluded since they only contain the tiled images for an associated DZI metadata file and a metadata.txt, if they were converted by the CS\footnote{	Compare chapter \ref{sec3_cs}
%} (see line 7 in \texttt{files{\textunderscore}from{\textunderscore}dir(dir)}).

\section{WSI}

\subsubsection{wsi(file)}

\begin{lstlisting}[frame=single,language=python]
def wsi(file):
	slide = OpenSlide(file)
	slide_name = file.split('/')[-1]
	regions = read_json(file + '_' + DICTIONARY)
	
	for region in regions:
		if(TESSELLATE):
			tessellate_wsi(slide, slide_name, region)
		else:
			bounding_box = get_bounding_box(region)
			if(RESIZE):
				bounding_box = resize_bounding_box(bounding_box)
			location = (bounding_box['x_min'], bounding_box['y_min'])
			size = (bounding_box['x_max'] - bounding_box['x_min'], bounding_box['y_max'] - bounding_box['y_min'])
			image = slide.read_region(location, 0, size)
			save_image(image, region, slide_name)
	
	slide.close()
\end{lstlisting}


\subsubsection{tessellate{\textunderscore}wsi(slide, slide{\textunderscore}name, region)}

\begin{lstlisting}[frame=single,language=python]
def tessellate_wsi(slide, slide_name, region):
	n,m = slide.dimensions
	m = m / TESSELLATE[HEIGHT]
	n = n / TESSELLATE[WIDTH]
	
	if SHOW:
		ox = 999999
		oy = 999999
	
	contour = []
	for coords in region.get('imgCoords'):
	if SHOW:
		if(coords.get('y') < oy): oy = coords.get('y')
		if(coords.get('x') < ox): ox = coords.get('x')
	x = int(coords.get('x') / TESSELLATE[WIDTH])
	y = int(coords.get('y') / TESSELLATE[HEIGHT])
	if [x, y] not in contour:
		contour.append([x, y])
	
	contour = np.asarray(contour)
	ref_img = Image.new('RGB', (n,m))
	cv_ref_img = np.array(ref_img)
	cv2.drawContours(cv_ref_img, [contour], 0, (255,255,255), -1)
	if SHOW:
		dbg_img = Image.new('RGB', (n,m))
	tiles = []
	for i in xrange(0, m):
		for j in xrange(0, n):
			px = cv_ref_img[i,j]
			if (px == [255, 255, 255]).all():
				location = ((j) * TESSELLATE[WIDTH], (i) * TESSELLATE[HEIGHT])
				size = TESSELLATE
				tile = slide.read_region(location, 0, size)
				tile_name = save_image(tile, region, slide_name, i, j)
				tiles.append(tile_name.split('/')[-1] + '.jpeg')
				if SHOW:
					dbg_img.paste(tile, (j * TESSELLATE[WIDTH] - int(ox), i * TESSELLATE[HEIGHT] - int(oy)))
	if SHOW:
	dbg_img.show()
	save_metadata(generate_file_name(region, slide_name), region, tiles)
\end{lstlisting}


\subsubsection{is{\textunderscore}supported(file)}

\begin{lstlisting}[frame=single,language=python]
def is_suppoted(file):
	ext = (file.split('.'))[-1]
	if(
		'bif' in ext or
		'mrxs' in ext or
		'npdi' in ext or
		'scn' in ext or
		'svs' in ext or
		'svslide' in ext or
		'tif' in ext or
		'tiff' in ext or
		'vms' in ext or
		'vmu' in ext
	):
		return 1
	else:
		return 0
\end{lstlisting}


\section{DZI}

\subsubsection{dzi(file)}

\begin{lstlisting}[frame=single,language=python]
def dzi(file):
	slide_name = file.split('/')[-1]
	with open(file, 'r') as dzi_file:
		content = dzi_file.read()
	root = ET.fromstring(content)
	dzi = {'tile_size': int(root.get('TileSize')), 'width': int(root[0].get('Width')), 'height': int(root[0].get('Height')), 'tile_source': get_tile_source(file), 'format': root.get('Format')}
	regions = read_json(file + '_' + DICTIONARY)
	
	for region in regions:
		if TESSELLATE:
			tessellate_dzi(dzi, slide_name, region)
		else:
		bounding_box = get_bounding_box(region)
		image = create_image_from_tiles(dzi, bounding_box)
		save_image(image, region, slide_name)
\end{lstlisting}


\subsubsection{create{\textunderscore}image{\textunderscore}from{\textunderscore}tiles(dzi, bounding{\textunderscore}box)}

\begin{lstlisting}[frame=single,language=python]
def create_image_from_tiles(dzi, bounding_box):
	if(RESIZE):
		bounding_box = resize_bounding_box(bounding_box)
	tile_image = get_tiles_from_bounding_box(dzi, bounding_box)
	
	offset_x = bounding_box['x_min']
	offset_y = bounding_box['y_min']
	
	x_min = bounding_box['x_min'] - offset_x
	x_max = bounding_box['x_max'] - offset_x
	y_min = bounding_box['y_min'] - offset_y
	y_max = bounding_box['y_max'] - offset_y
	
	return tile_image.crop((x_min, y_min, x_max, y_max))
\end{lstlisting}


\subsubsection{get{\textunderscore}tiles{\textunderscore}from{\textunderscore}bounding(dzi, bounding{\textunderscore}box)}

\begin{lstlisting}[frame=single,language=python]
def get_tiles_from_bounding_box(dzi, bounding_box):
	x_min = bounding_box['x_min'] / dzi['tile_size']
	x_max = bounding_box['x_max'] / dzi['tile_size']
	y_min = bounding_box['y_min'] / dzi['tile_size']
	y_max = bounding_box['y_max'] / dzi['tile_size']
	
	stitch = Image.new('RGB', ((x_max-x_min+1) * dzi['tile_size'], (y_max-y_min+1) * dzi['tile_size']))
	
	for i in range(x_min, x_max+1):
		for j in range(y_min, y_max+1):
			tile = Image.open(dzi['tile_source'] + str(i) + '_' + str(j) + '.' + dzi['format'])
			stitch.paste(tile, ((i - x_min) * dzi['tile_size'], (j - y_min) * dzi['tile_size']))
	return stitch
\end{lstlisting}


\subsubsection{get{\textunderscore}tile{\textunderscore}source(file)}

\begin{lstlisting}[frame=single,language=python]
def get_tile_source(file):
	files_dir = file.replace('.dzi', '_files/')
	layers = os.listdir(files_dir)
	layers.remove('metadata.txt')
	layers = map(int, layers)
	return files_dir + str(max(layers)) + '/'
\end{lstlisting}


\subsubsection{tessellate{\textunderscore}dzi(dzi, slide{\textunderscore}name, region)}

\begin{lstlisting}[frame=single,language=python]
def tessellate_dzi(dzi, slide_name, region):
	bounding_box = get_bounding_box(region)
	tile_image = get_tiles_from_bounding_box(dzi, bounding_box)
	
	offset_x = bounding_box['x_min']
	offset_y = bounding_box['y_min']
	
	n,m = tile_image.size
	
	m = m / TESSELLATE[HEIGHT]
	n = n / TESSELLATE[WIDTH]
	
	contour = []
	for coords in region.get('imgCoords'):
	x = int((coords.get('x') - offset_x) / TESSELLATE[WIDTH])
	y = int((coords.get('y') - offset_y) / TESSELLATE[HEIGHT])
	if [x, y] not in contour:
		contour.append([x, y])
	
	contour = np.asarray(contour)
	ref_img = Image.new('RGB', (n,m))
	cv_ref_img = np.array(ref_img)
	cv2.drawContours(cv_ref_img, [contour], 0, (255,255,255), -1)
	if SHOW:
		dbg_img = Image.new('RGB', tile_image.size)
	tiles = []
	for i in xrange(0, m):
		for j in xrange(0, n):
			px = cv_ref_img[i,j]
			if (px == [255, 255, 255]).all():
				tile = tile_image.crop((j * TESSELLATE[WIDTH] + (bounding_box['x_min'] % dzi['tile_size']),
				i * TESSELLATE[HEIGHT] + (bounding_box['y_min'] % dzi['tile_size']),
				j * TESSELLATE[WIDTH] + (bounding_box['x_min'] % dzi['tile_size']) + TESSELLATE[WIDTH],
				i * TESSELLATE[HEIGHT] + (bounding_box['y_min'] % dzi['tile_size']) + TESSELLATE[HEIGHT]))
				tile_name = save_image(tile, region, slide_name, i, j)
				tiles.append(tile_name.split('/')[-1] + '.jpeg')
				if SHOW:
					dbg_img.paste(tile, (j * TESSELLATE[WIDTH], i * TESSELLATE[HEIGHT]))
	if SHOW:
		dbg_img.show()
	save_metadata(generate_file_name(region, slide_name), region, tiles)
\end{lstlisting}


\section{Utility}

\subsubsection{read{\textunderscore}json(path)}

\begin{lstlisting}[frame=single,language=python]
def read_json(path):
	try:
		with open(path, 'r') as file:
		str = (file.read())
		data = json.loads(str.decode('utf-8'))
		return data
	except IOError:
		print('Could not load saved annotations from ' + path)
\end{lstlisting}


\subsubsection{save{\textunderscore}metadata(name, region, *tiles)}

\begin{lstlisting}[frame=single,language=python]
def save_metadata(name, region, *tiles):
	if len(tiles) > 0:
	name = name + '_tessellated.metadata.json'
	if not FORCE:
		cnt = 0
		while os.path.isfile(name):
			cnt+=1
		name = name + '(' + str(cnt) +')'
	else:
		image_name = name
	name = name + '.metadata.json'
	with open(name, 'w+') as file:
		data = {'label': region.get('name'), 'zoom': region.get('zoom'), 'context': region.get('context')}
	if len(tiles) > 0:
		data['tiles'] = tiles
	else:
		data['image'] = image_name.split('/')[-1] + '.jpeg'
	content = json.dumps(data, ensure_ascii=False)
	file.write(content.encode('utf-8'))
\end{lstlisting}


\subsubsection{generate{\textunderscore}file{\textunderscore}name(region, slide{\textunderscore}name, *tiles)}

\begin{lstlisting}[frame=single,language=python]
def generate_file_name(region, slide_name, *tiles):
	if(OUTPUT):
		dest = OUTPUT + region['name']
	else:
		dest = region['name']
	if not os.path.exists(dest):
		os.makedirs(dest)
	name = dest + '/' + slide_name + '_' + str(region['uid'])
	if len(tiles) > 0:
		for entry in tiles:
			name += "_" + str(entry)
	if not FORCE:
		cnt = 0
		while os.path.isfile(name):
			cnt+=1
		name = name + '(' + str(cnt) +')'
	return name
\end{lstlisting}


\subsubsection{save{\textunderscore}image(image, region, slide{\textunderscore}name, *tiles}

\begin{lstlisting}[frame=single,language=python]
def save_image(image, region, slide_name, *tiles):
	if len(tiles) == 0:
		name = generate_file_name(region, slide_name)
	else:
		name = generate_file_name(region, slide_name, tiles)
	if RESIZE:
		image = image.resize(RESIZE, INTERPOLATION)
	# L = R * 299/1000 + G * 587/1000 + B * 114/1000
	if GRAYSCALE:
		image = image.convert('L')
	image.save(name + '.jpeg', 'jpeg')
	if len(tiles) == 0:
		save_metadata(name, region)
	return name
\end{lstlisting}


\subsubsection{get{\textunderscore}bounding{\textunderscore}box(region)}

\begin{lstlisting}[frame=single,language=python]
def get_bounding_box(region):
	x_min = sys.float_info.max
	x_max = sys.float_info.min
	y_min = x_min
	y_max = x_max
	for coordinate in region.get('imgCoords'):
		x = coordinate.get('x')
		y = coordinate.get('y')
		if(x >= x_max):
			x_max = x
		if(x < x_min) :
			x_min = x
		if(y >= y_max):
			y_max = y
		if(y < y_min) :
			y_min = y
	
	return {'x_max': int(np.ceil(x_max)), 'x_min': int(np.floor(x_min)),
		'y_max': int(np.ceil(y_max)), 'y_min': int(np.floor(y_min))}
\end{lstlisting}


\subsubsection{resize{\textunderscore}bounding{\textunderscore}box(region)}

\begin{lstlisting}[frame=single,language=python]
def resize_bounding_box(bounding_box):
	r_ratio = RESIZE[WIDTH] / float(RESIZE[HEIGHT])
	bb_width = float(bounding_box['x_max'] - bounding_box['x_min'])
	bb_height = float(bounding_box['y_max'] - bounding_box['y_min'])
	bb_ratio = bb_width / bb_height
	if r_ratio == bb_ratio:
		return bounding_box
	else:
		if r_ratio == 1:
			# target is square
			s1 = bb_height/bb_width
			s2 = bb_width/bb_height
			scaled = min(bb_width, bb_height) * max(s1, s2) - min(bb_width, bb_height)
			if(bb_width > bb_height):
				bounding_box['y_min'] -= int(np.floor(scaled/2))
				bounding_box['y_max'] += int(np.ceil(scaled/2))
			else:
				bounding_box['x_min'] -= int(np.floor(scaled/2))
				bounding_box['x_max'] += int(np.ceil(scaled/2))
		elif r_ratio < 1:
			# target is higher than wide
			h_s = 1 / r_ratio
			if bb_height > (bb_width * h_s):
				# adjust width:
				w_new = (bb_height / h_s) - bb_width
				bounding_box['x_min'] -= int(np.floor(w_new/2))
				bounding_box['x_max'] += int(np.ceil(w_new/2))
			else:
				# adjust height:
				h_new = h_s * bb_width - bb_height
				bounding_box['y_min'] -= int(np.floor(h_new/2))
				bounding_box['y_max'] += int(np.ceil(h_new/2))
		else:
			# target is wider than high
			w_s = r_ratio
			if bb_width > (bb_height * w_s):
				# adjust height
				h_new = (bb_width / w_s) - bb_height
				bounding_box['y_min'] -= int(np.floor(h_new/2))
				bounding_box['y_max'] += int(np.ceil(h_new/2))
			else:
				# adjust width:
				w_new = w_s * bb_height - bb_width
				bounding_box['x_min'] -= int(np.floor(w_new/2))
				bounding_box['x_max'] += int(np.ceil(w_new/2))
		
		# check if bb is big enough
		bb_width = float(bounding_box['x_max'] - bounding_box['x_min'])
		if bb_width < RESIZE[WIDTH]:
			s = RESIZE[WIDTH] / bb_width
			bounding_box = scale_bounding_box(bounding_box, s)
		bb_height = float(bounding_box['y_max'] - bounding_box['y_min'])
		if bb_height < RESIZE[HEIGHT]:
			s = RESIZE[HEIGHT] / bb_height
			bounding_box = scale_bounding_box(bounding_box, s)
		
		if(bounding_box['y_min'] < 0):
			dif = bounding_box['y_min'] * (-1)
			bounding_box['y_min'] += dif
			bounding_box['y_max'] += dif
		if(bounding_box['x_min'] < 0):
			dif = bounding_box['x_min'] * (-1)
			bounding_box['x_min'] += dif
			bounding_box['x_max'] += dif
	
		return bounding_box
\end{lstlisting}


\subsubsection{scale{\textunderscore}bounding{\textunderscore}box(bounding{\textunderscore}, scale)}

\begin{lstlisting}[frame=single,language=python]
def scale_bounding_box(bounding_box, scale):
	bb_width = float(bounding_box['x_max'] - bounding_box['x_min'])
	add_w = (bb_width * scale) - bb_width
	bounding_box['x_min'] -= int(np.floor(add_w/2))
	bounding_box['x_max'] += int(np.ceil(add_w/2))
	
	bb_height = float(bounding_box['y_max'] - bounding_box['y_min'])
	add_h = (bb_height * scale) - bb_height
	bounding_box['y_min'] -= int(np.floor(add_h/2))
	bounding_box['y_max'] += int(np.ceil(add_h/2))
	
	return bounding_box
\end{lstlisting}