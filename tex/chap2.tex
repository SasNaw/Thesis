\chapter{Background}
\section{Definition of terms}
\subsection{Deep Zoom Image Format}
The Deep Zoom Image Format (.dzi)\nomenclature{.dzi}{Deep Zoom Image Format} is an xml-based file format maintained by Microsoft to improve performance and quality in the handling of large image files. Therefore an image will be represented in a tiled pyramid scheme (see fig. 2.1).
			
	\begin{figure}[!htbp]
		\begin{center}
			\includegraphics[scale=0.5]{img/dzi_pyramid.png}
				\caption{of the dzi pyramid image representation (source: https://i-msdn.sec.s-msft.com/dynimg/IC141135.png)}
			\label{fig:abb2.1}
		\end{center}
	\end{figure}

As seen in fig. 2.1 there are multiple versions of a single image in different resolutions. The idea behind this is, that if a user wants to see a whole picture zoomed out or as a small thumbnail, it is not necessary to load the image file in its highest resolution. To save bandwidth a version with a smaller resolution is loaded. If the user wishes to zoom in on a specific area of the image, a version with a higher resolution is loaded. Once again, however, it is not necessary to load the whole image, since only a fraction of it will be visible. For this reason there are tiles of the image which are loaded instead (see highlited tiles in fig. 2.1) \cite{web:dzi}.

Each resolution in the pyramid is called a \emph{level}. At each level the image is scaled down by the factor 4 (2 in each dimension). In other words, a level can be defined as an image with the resolution 2*level for height and width, resulting in a resolution of (2*level)*(2*level). Levels are counted from level 0 (1*1 Pixel) \cite{web:dzi}. E.g. the levels shown in fig. 2.1 are:
\begin{itemize}
	\item level 8 ($2^8=256$) for the $256^2$ pixel image
	\item level 9 ($2^9=512$) for the $512^2$ pixel image
	\item level 10 ($2^10=1024$) for the $1024^2$ pixel image
\end{itemize}

\subsection{Microservice}
\subsection{Machine Learning}
\subsection{Neural Networks}
\section{Process chain}
\subsection{Description}
\subsection{Definition of Conversion Service}
\subsection{Definition of Annotation Service}
\subsection{Definition of Tesselation Service}