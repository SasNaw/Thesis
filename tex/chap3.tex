\chapter{Conversion Service}
\label{sec3_cs}
\section{Methodology}
As stated in section \ref{sec2_wsif}, there is no standardized format for WSIs. Supplement 145 of the DICOM standard tries to unify the whole process around WSIs, but vendors still push their proprietary formats. For the reasons mentioned in subsection \ref{sec2_openFormats}, it is necessary to establish a common format for all the WSIs which are subject to the process chain established in section \ref{sec2_pc}. Therefore, the goal of the CS is to convert WSIs of proprietary formats into a common open format.

To make the conversion as convenient and fast as possible, the CS should only have brief user interaction. For this purpose it will not have a GUI. Instead the CS will be implemented as a console script. Furthermore, the CS should be capable of converting multiple WSIs after one another, so that no restart is necessary between conversions. Therefore, the CS will take an input directory as parameter and convert all WSIs of valid format inside that directory. Another parameter will be the output folder, in which the converted DZIs are stored.

\begin{table}[H]
	\begin{center}
		\begin{tabular}{| l | l |}
			\hline
			\textbf{vendor} & \textbf{formats}\\ \hline
			Aperio & SVS, TIF\\ \hline
			Hamamatsu & VMS, VMU, NDPI\\ \hline
			Leica & SCN\\ \hline
			3DHistech/Mirax & MRXS\\ \hline
			Philips & TIFF\\ \hline
			Sakura & SVSLIDE\\ \hline
			Trestle & TIF\\ \hline
			Ventana & BIF, TIF\\ \hline
		\end{tabular}
		\caption{File formats by vendor}
		\label{tab3_formatsByVendor}
	\end{center}
\end{table}

Tab. \ref{tab3_formatsByVendor} gives an overview of file formats, sorted by vendor, which are viable as input for the conversion.


\subsection{Selection of Image Format}
A format or service must be chosen as conversion target for the CS. Choices have been established in \ref{sec2_openFormats}. These are:
\begin{enumerate}[(1)]
	\item BigTIFF
	\item DZI
	\item IFFF
	\item JPEG 2000
	\item TMS/OMS
\end{enumerate}

To convert a WSI, a conversion tool is needed. Tab. \ref{tab3_convOptions} shows a listing of possibilities for that purpose. Listed are the name of the tool, the technology used and the output format. The table indicates, that DZI has a great variety of options, while the alternatives have little to none (Map Tiler for TMS, Kakadu for IFFF and none for the others).

\begin{longtable}{| p{3cm} | p{4cm} | p{2.5cm} |}
	\hline
	Deep Zoom Composer & dekstop app for Windows & DZI \\ \hline
	Image Composite Editor & panoramic image stitcher from Microsoft Research for the Windows desktop & DZI \\ \hline
	DeepZoomTools.dll & .NET-library, comes with Deep Zoom Composer & DZI \\ \hline
	deepzoom.py & Python & DZI \\ \hline
	deepzoom & Perl utility & DZI \\ \hline
	PHP Deep Zoom Tools & PHP & DZI \\ \hline
	Deepzoom & PHP & DZI \\ \hline
	DZT & an image slicing library and tool written in Ruby & DZI \\ \hline
	MapTiler &  desktop app for Windows, Mac, Linux & TMS \\ \hline
	VIPS & command line tool and library for a number of languages & DZI \\ \hline
	Sharp & Node.js, uses VIPS & DZI \\ \hline
	MagickSlicer & shell script (Linux/Mac) & DZI \\ \hline
	Gmap Uploader Tiler & C++ & DZI \\ \hline
	Node.js Deep Zoom Tools & Node.js, under construction & DZI \\ \hline
	OpenSeaDragon DZI Online Composer & Web app (and PERL and PHP scripts) & DZI \\ \hline
	Zoomable & service, offers embeds; no explicit API & DZI \\ \hline
	ZoomHub & service, under construction & DZI \\ \hline
	Kakadu & C++ library to encode or decode JPEG 2000 images & IIIF \\ \hline
	PyramidIO & Java (command line and library) & DZI \\ \hline
	\caption{Overview of conversion options for zooming image formats (source: \cite{web:openseadragon})}
	\label{tab3_convOptions}
\end{longtable}

Since the CS should only consist of brief user interaction and be as automated as possible, desktop and web applications are not valid as tools for conversion. This excludes \emph{Deep Zoom Composer}, \emph{MapTiler}, \emph{OpenSeaDragon DZI Online Composer} and \emph{Zoomable} as possible choices (therefore also excluding (next to the reasons given in subsection \ref{sec2_openFormats}), TMS as possible format).

One of the reasons not to use proprietary formats was the support of only certain operating systems, eliminating Windows-only tools. Those are \emph{Image Composite Editor} and \emph{DeepZoomTools.dll}.

Furthermore, reading the proprietary formats is a highly specialized task, eliminating most of the leftover choices: \emph{deepzoom}\cite{web:deepzoom}, \emph{DZT}\cite{web:dzt}, \emph{sharp}\cite{web:sharp}, \emph{MagickSlicer}, \emph{Node.js Deep Zoom Tools}\footnote{MagickSlicer and Node.js Deep Zoom Tools use ImageMagick to read images, which does not support any of the proprietary WSI formats\cite{web:imagemagick}.}, \emph{Gmap Uploader Tiler}\cite{web:gmap}, Zoomhub\cite{web:zoomhub} and \emph{PyramidIO}\cite{web:pyramidio}.

\emph{Kakadu} can only encode and decode JPEG 2000 images\cite{web:openseadragon}, making it no valid choice either.

This leaves \emph{deepzoom.py} and \emph{VIPS}, both creating DZI as output. Through the use of OpenSlide, they are both capable of reading all proprietary formats stated in tab. \ref{tab3_formatsByVendor}\cite{web:openslide}.


\subsection{Deepzoom.py}

Deepzoom.py\footnote{See \url{https://github.com/openzoom/deepzoom.py} for further details} is a python script and part of Open Zoom\footnote{See \url{https://github.com/openzoom} for further details}. It can either be called directly over a terminal or imported as a module in another python script. The conversion procedure itself is analogous for both methods.

If run in a terminal the call looks like the following:

\begin{lstlisting}
	$ python deepzoom.py [options] [input file]
\end{lstlisting}

The various options and their default values can be seen in \ref{tab3_deepzoomParams}. If called without a designated output destination, deepzoom.py will save the converted DZI in the same directory as the input file.

\begin{table}[H]
	\begin{center}
		\begin{tabular}{| r | l | r |}
			\hline
			\textbf{option} & \textbf{description} & \textbf{default} \\ \hline
			-h & show help dialog & - \\ \hline
			-d & output destination & - \\ \hline
			-s & size of the tiles in pixels & 254 \\ \hline
			-f & image format of the tiles & jpg\\ \hline
			-o & overlap of the tiles in pixels (0 - 10) & 1 \\ \hline
			-q & quality of the output image (0.0 - 1.0) & 0.8 \\ \hline
			-r & type of resize filter & antialias \\ \hline
		\end{tabular}
		\caption{Options for deepzoom.py}
		\label{tab3_deepzoomParams}
	\end{center}
\end{table}

The resize filter is applied to interpolate the pixels of the image when changing its size for the different levels. Supported filters are:

\begin{itemize}
	\item cubic
	\item biliniear
	\item bicubic
	\item nearest
	\item antialias
\end{itemize}

When used as module in another python script, deepzoom.py can simply be imported via the usual \emph{import} command. To actually use deepzoom.py, a Deep Zoom Image Creator needs to be created. This class will manage the conversion process:

\begin{lstlisting}[frame=single]
# Create Deep Zoom Image Creator
creator = deepzoom.ImageCreator(tile_size=[size], 
	tile_overlap=[overlap],	tile_format=[format], 
	image_quality=[quality], resize_filter=[filter])
\end{lstlisting}

The options are analogous with the terminal version (compare tab. \ref{tab3_deepzoomParams}). To start the conversion process, the following call must be made within the python script:

\begin{lstlisting}[frame=single]
# Create Deep Zoom image pyramid from source
creator.create([source], [destination])
\end{lstlisting}

In the proposed workflow, the ImageCreator opens the input image $img^{wsi}$ and accesses the information necessary to create the describing XML file for the DZI\footnote{Compare subsection \ref{sec2_openFormats}}. The needed number of levels is calculated next. For this, the bigger value of height or width of $img^{wsi}$ is chosen (see eq. 3.1) and then used to determine the number of levels $lvl^{max}$ (see eq. 3.2) necessary.

\begin{equation}
	max{\_}dim = max(height, width)
\end{equation}

\begin{equation}
	lvl^{max} = {\lceil}log_2(max{\_}dim) + 1\rceil
\end{equation}

Once $lvl^{max}$ has been determined, a resized version $img^{dzi}_i$ of $img^{wsi}$ will be created for every level $i \in [0, lvl-1]$. The quality of $img^{dzi}_i$ will be reduced according to the value specified for -q/image{\textunderscore}quality (see tab. \ref{tab3_deepzoomParams}). The resolution of $img^{dzi}_i$ will be calculated with the \emph{scale} function (see eq. 3.3) for both, height and width. Furthermore, the image will be interpolated with the specified filter (-r/resize{\textunderscore}filter parameter, see tab. \ref{tab3_deepzoomParams}).

\begin{equation}
	scale = {\lceil}dim * 0.5^{lvl^{max}-i}\rceil
\end{equation}

Once $img^{dzi}_i$ has been created, it will be tessellated into as many tiles of the specified size (-s/tile{\textunderscore}size parameter, see tab. \ref{tab3_deepzoomParams}) and overlap (-o/tile{\textunderscore}overlap parameter, see tab. \ref{tab3_deepzoomParams}) as possible. If the size of $img^{wsi}$ in either dimension is not a multiple of the tile size, the last row/column of tiles will be smaller by the amount of $(tile$ $size - ([height$ or $width] \mod tile$ $size))$ pixels.

Every tile will be saved as [column]{\textunderscore}[row].[format] (depending on the -f/file{\textunderscore} format parameter, see tab. \ref{tab3_deepzoomParams}) in a directory named according to the corresponding level $i$. Each one of those level directories will be contained within a directory called [filename]{\textunderscore}files. The describing XML file will be persisted as [filename].dzi in the same directory as [filename]{\textunderscore}files.


\subsection{VIPS}

VIPS (VASARI Image Processing System)\nmc{VIPS}{VASARI Image Processing System} is described as "[...] a free image processing system [...]" \cite{web:vips}. It includes a wide range of different image processing tools, such as various filters, histograms, geometric transformations and color processing algorithms. It also supports various scientific image formats, especially from the histopathological sector\cite{web:vips}. One of the strongest traits of VIPS is its speed and little data usage compared to other imaging libraries\cite{cupitt05}.

VIPS consists of two parts: the actual library (called libvips) and a GUI (called nip2). libvips offers interfaces for C, C++, python and the command line. The GUI will not be further discussed, since it is of no interest for the implementation of the CS. 

VIPS speed and little data usage is achieved by the usage of a fully demand-driven image input/output system. While conventional imaging libraries queue their operations and go through them sequentially, VIPS awaits a final write command, before actually manipulating the image. All the queued operations will then be evaluated and merged into a few single operations, requiring no additional disc space for intermediates and no unnecessary disc in- and output. Furthermore, if more than one CPU is available, VIPS will automatically evaluate the operations in parallel\cite{cupitt96}.

As mentioned before, VIPS has a command line and python interface. In either case, a function called \emph{dzsave} will manage the conversion from a WSI to a DZI. A call in the terminal looks as follows:

\begin{lstlisting}
$ vips dzsave [input] [output] [options]
\end{lstlisting}

When called, VIPS will take the image [input], convert it into a DZI and then save it to [output]. The various options and their default values can be seen in tab. \ref{tab3_optionsVips}.

\begin{table}[H]
	\begin{center}
		\begin{tabular}{| l | l | l |}
			\hline
			\textbf{option} & \textbf{description} & \textbf{default} \\ \hline
			layout & directory layout (allowed: dz, google, zoomify) & dz \\ \hline
			overlap & tile overlap in pixels & 1 \\ \hline
			centre & center image in tile & false \\ \hline
			depth & pyramid depth & onepixel \\ \hline
			angle & rotate image during save & d0 \\ \hline
			container & pyramid container type & fs \\ \hline
			properties & write a properties file to the output directory & false \\ \hline
			strip & strip all metadata from image & false \\ \hline
		\end{tabular}
		\caption{Options for VIPS}
		\label{tab3_optionsVips}
	\end{center}
\end{table}

A call in python has the same parameters and default values. It looks like this:

\begin{lstlisting}[frame=single]
image = Vips.Image.new_from_file(input)
image.dzsave(output[, options])
\end{lstlisting}

In line 1 the image gets opened and saved into a local variable called \emph{image}. While being opened, further operations could be done. The command in line 2 writes the processed image as DZI into the specified output location.


\section{Implementation}

As stated before, the CS should be implemented as a script.

The first iteration was a python script using deepzoom.py for the conversion. This caused severe performance issues. Out of all the image files in the test set\footnote{See section \ref{sec3_test}}, only one could be converted\footnote{CMU-3.svs from Aperio, see appendix \ref{sec_A1}}. Other files were either too big, so the process would eventually be killed by the operating system, or exited with an IOError concerning the input file from the PIL imaging library.

The second iteration uses VIPS python implementation, which is capable of converting all the given test images.

The script has to be called inside a terminal in the following fashion:

\begin{lstlisting}
	$ python ConversionService.py [input dir] [output dir]
\end{lstlisting}

Both the input and the output directory parameter are mandatory, in order for the script to know where to look for images to convert and where to save the resulting DZIs.

Upon calling, the \emph{main()} routine will be started, which orchestrates the whole conversion process. The source code is as follows:

\begin{lstlisting}[frame=single]
def main():
	path = checkParams()
	files = os.listdir(path)
	for file in files:
		print("-----------------------------------------")
		extLen = getFileExt(file)
		if(extLen != 0):
			print("converting " + file + "...")
			convert(path, file, extLen)
			print("done!")
\end{lstlisting}

\emph{checkParams()} checks if the input parameters are valid and, if so, returns the path to the specified folder or aborts the execution otherwise. Furthermore, it will create the specified output folder, if it does not exist already. In the next step, the specified input folder will be checked for its content. \emph{getFileExt(file)} looks up the extension of each contained file and will either return the length of the files extension or $0$ otherwise. Each valid file will then be converted with the \emph{convert(...)} function:

\begin{lstlisting}[frame=single]
# convert image source into .dzi format and copies all header 
# information into [img]_files dir as metadata.txt
# param path: directory of param file
# param file: file to be converted
# param extLen: length of file extension
def convert(path, file, extLen):
	dzi = OUTPUT + file[:extLen] + "dzi"
	im = Vips.Image.new_from_file(path + file)
	# get image header and save to metadata file
	im.dzsave(dzi, overlap=OVERLAP, tile_size=TILESIZE)
	# create file for header
	headerOutput = OUTPUT + file[:extLen-1] + "_files/metadata.txt"
	bashCommand = "touch " + headerOutput
	call(bashCommand.split())
	# get header information
	bashCommand = "vipsheader -a " + path + file
	p = subprocess.Popen(bashCommand.split(), stdout=subprocess.PIPE, stderr=subprocess.PIPE)
	out, err = p.communicate()
	# write header information to file
	text_file = open(headerOutput, "w")
	text_file.write(out)
	text_file.close()
\end{lstlisting}

The name for the new DZI file will be created from the original file name, however, the former extension will be replaced by "dzi" (see line 7). \emph{OUTPUT} specifies the output directory which the file will be saved to. Next, the image file will be opened with Vips' Image class. Afterwards, \emph{dzsave(...)} will be called, which handles the actual conversion into the dzi file format. \emph{OVERLAP} and \emph{TILESIZE} are global variables which describe the overlap of the tiles and their respective size. Their values are 0 (\emph{OVERLAP}) and 256 (\emph{TILESIZE}). The output will be saved to the current working directory of ConversionService.py, appending "/dzi/[\emph{OUTPUT}]/".

When a WSI gets converted into DZI by the CS, most of the image header information is lost. To counteract this, a file \emph{metadata.txt} is created in the [name]{\textunderscore}files directory, which serves as container for the header information of the original WSI (see line 12 and 13).

The console command \emph{vipsheader -a} is responsible for extracting the header information (see line 17 - 18). The read information (\emph{out} in line 18) is then written into the \emph{metadata.txt} file (Line 20 - 22).


\section{Test}
\label{sec3_test}
To test the correct functionality of the CS a test data set was needed. OpenSlide offers a selection of freely distributable WSIs\footnote{See OpenSlides Homepage: \url{http://openslide.cs.cmu.edu}, or directly for the test data: \url{http://openslide.cs.cmu.edu/download/openslide-testdata/}}, which can be used for that purpose.

Since the size of the WSIs is big, they are not delivered via the CS repository\footnote{See \url{https://github.com/SasNaw/ConversionService}}. Instead they need to be downloaded separately from the OpenSlide homepage. For a complete listing of the used test data see Appendix A.


\subsection{Setup}

To create a controlled environment for the test, a new directory will be created, called \emph{CS{\textunderscore}test}. A copy of ConversionService.py as well as a directory containing all the test WSIs (called \emph{input}) will be placed in that directory.

\emph{Input} contains the following slides:

\begin{enumerate}[(1)]
	\item CMU-2 (Aperio, .svs)
	\item CMU-1 (Generic Tiled tiff, .tiff)
	\item OS-3 (Hamamatsu, .ndpi)
	\item CMU-2 (Hamamatsu, .vms)
	\item Leica-2 (Leica, .scn)
	\item Mirax2.2-3 (Mirax, .mrxs)
	\item CMU-2 (Trestle, .tif)
	\item OS-2 (Ventana, .bif)
\end{enumerate}

Because of their structure, (4), (6) and (7) will be placed in directories titled with their file extension. Fig. \ref{fig3_content} shows the content of the input folder.

\begin{figure}[H]
	\begin{center}
		\includegraphics[scale=0.5]{img/inputDir.png}
		\caption{Content of input directory}
		\label{fig3_content}
	\end{center}
\end{figure}

This makes multiple calls of the CS necessary. The calls, in that order, are:

\begin{lstlisting}
	$ python ConversionService.py input/ out_1/
	$ python ConversionService.py input/mrxs out_2/
	$ python ConversionService.py input/tif out_3/
	$ python ConversionService.py input/vms out_4/
\end{lstlisting}


\subsection{Result}

All runs of ConversionService.py were successful. Tab. \ref{tab3_results} shows an overview of the results:

\begin{table}[H]
	\begin{center}
		\begin{tabular}{| p{2cm} | p{6cm} | R{2cm} |}
			\hline
			\textbf{input} & \textbf{output} & \textbf{time (sec)} \\ \hline
			input/ & CMU-1.dzi, CMU-2.dzi, Leica-2.dzi, OS-2.dzi, OS-3.dzi & 1992 \\ \hline
			input/mrxs/ & Mirax2.2-3.dzi & 500 \\ \hline
			input/tif/ & CMU-2.dzi & 56 \\ \hline
			input/vms/ & CMU-2-40x - 2010-01-12 13.38.58.dzi & 305 \\ \hline
		\end{tabular}
		\caption{Results of Conversion Service Test}
		\label{tab3_results}
	\end{center}
\end{table}

The vast difference in file size of the test data accounts for the different run times of the tests. While the first test converted 5 WSIs (399 sec/WSI), every other test converted a single one. The conversion of (6) was much faster, since the file was smaller in size (304.22 MB) compared to the others (1495.24 MB on average).