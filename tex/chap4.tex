\chapter{Annotation Service}

\section{Objective of the Annotation Service}
\label{sec4_objective}
As described in \ref{sec2_as}, the goal of the AS is to provide a user with the possibility to:
\begin{enumerate}[(1)]
	\item view A WSI
	\item annotate a WSI
	\item manage made annotations
\end{enumerate}

In order to achieve objective (1) - (3), a GUI needs to be deployed which supports the user in working on those tasks. (3) also adds the need for file persistence management.

It became clear during the development process that the support of only DZI was impractical for the real life environment of the AS, thus making it necessary to support proprietary formats as well. A solution has to be found, that still addresses the vendor and platform issues stated in \ref{sec1_researchObjective} and \ref{sec2_openFormats}.


\section{Functionality of the Annotation Service}
\label{sec4_functions}
The goal of viewing a WSI (1) is a straight forward task. (2) and (3) are more elusive. For that reason, this subsection elaborates on the functionality needed to help achieve those objectives.

Annotations will be created by drawing directly onto the viewed WSI. If the user spots a region of interest, a contour can be drawn around it. This can  either be done in \emph{free hand} or \emph{polygon mode}. In free hand mode, the contour will be drawn along the path of the mouse pointer, until the mode is disabled again. Upon deactivation, the contour will be closed. In polygon mode, the user can place coordinates which will be connected from one to another in the order they are placed in. A contour in this mode considers to be closed, once a point on the contour is clicked a second time. A marked region of interest is simply called \emph{region} from this point on.

The information what a region is surrounded by can be as valuable as the information about the region itself\cite{Bankman00}. Therefore, every region will have a \emph{context} trait, which lists every label of regions it touches, crosses, surrounds or is surrounded by (see fig. \ref{fig4_contextregions}).

\begin{figure}[H]
	\begin{center}
		\includegraphics[scale=0.5]{img/contextregions.png}
		\caption{Example of context regions (B, C are context of A; A, C are context of B; A, B are context of C; D has no context region)}
		\label{fig4_contextregions}
	\end{center}
\end{figure}
% TODO: add figure about context regions

Another way of creating regions will be so called \emph{points of interest} (POI)\nmc{POI}{Point of Interest}. A POI will be placed with a mouse click. After that, an external script will be invoked to run an automated segmentation in the proximity of the POI and return with a contour which will be marked as region. The segmentation approaches may differ drastically in different scenarios\cite{Liu12}, therefore the script will be interchangeable\footnote{There are many different ways of how to approach the topic of segmentation (e.g. \cite{Qi12}, \cite{Sharma16}, \cite{Wienert12}, \cite{Angulo10} for cell segmentation alone). Writing a fully working segmentation script is worth another thesis by itself, therefore only a dummy implementation will be delivered with this work.}.

Each region has a \emph{label} associated to it. A label is a predefined string, which describes what the region just created shows. The labels available will be determined through a \emph{label dictionary}, which is a container that offers a list of strings to select from. This approach guarantees a unified labeling, independent of a specific WSI or pathologist. The option to choose between multiple available label dictionaries opens up the possibility of creating dictionaries which are specialized on certain cases or studies. Again, to keep up labeling integrity, labels can only be selected from one dictionary per WSI.

Users will be able to create new, empty dictionaries, if the need arises. Furthermore, they will be able to add entries to existing and new dictionaries alike, to further advance or specialize them. To delete single entries or whole dictionaries, file access to the server is necessary. This is due to the fact, that knowledge can by added without direct negative consequences. Deleting existing knowledge influences all WSIs on which this knowledge was used, may it be as a label or a whole dictionary.

To support the user in annotating a WSI, a distance measurement tool will be usable as well. This tool can measure the distance between 2 pixels in $\mu$m.


\section{Methodology}
\label{sec4_methodology}
As stated in \ref{sec2_openFormats}, most vendors have proprietary image formats and their own implementation of a viewer for those, thus creating a vendor lock-in. Further do vendors often support only Windows platforms, ignoring other operating systems\cite{Cornish13},\cite{DICOM10},\cite{Farahanil15}. To avoid this, a solution must be found that is independent of operating system and vendor.

Independence from an operating system can be achieved by using web technologies, especially when running an application in a web browser, since those are supported by all modern operating systems and even mobile platforms\cite{Tseytlin14}.

To develop the AS as a web application means to become subject to \emph{cross-origin resource sharing} (CORS)\nmc{CORS}{Cross Origin Resource Sharing}\cite{Kesteren14} and the \emph{same-origin policy} (SOP)\cite{web:mdn}\nmc{SOP}{Same-Origin Policy}. The SOP is a security concept of the web application security model, that only allows direct file access if the parent directory of the originating file is an ancestor directory of the target file\cite{web:mdn}. Since the local WSI file will not have the same origin, CORS is needed. CORS is a standard that defines mechanisms to allow access to restricted resources from a domain outside of the origin, when using the HTTP protocol\cite{Kesteren14}. Since the WSI is a local file, HTTP can not be used to retrieve the file.

The restrictions of SOP and CORS can be worked around by deploying a server as so called \emph{digital slide repository} (DSR). A DSR manages storage of WSIs and their metadata\cite{Cornish13}. This way, WSIs would share the same origin as the viewer and their retrieval would be possible.

Using a DSR has additional advantages: 
\begin{itemize}
	\item WSIs are medical images and as such confidential information. Their access is usually tied to non-disclosure or confidentiality agreements (e.g. \cite{COA} or \cite{USSanDiego}). A DSR eliminates the need to hand out copies of WSIs, which makes it easier to uphold the mentioned agreements.
	\item WSIs take up big portions of storage\cite{Singh11}. The local systems used by pathologists in the environment of the AS are usual desktop computers and laptops. As such, their storage might be insufficient to hold data in those quantities. A DSR can be set up as a dedicated file server, equipped for the purpose of offering large amounts of storage.
	\item A DSR enables centralized file management. Pathologists don't access local their local version of a WSI and it's annotations, but share the same data pool.
	\item Depending on the network setup, other advantages become possible, e.g. sharing of rare cases as educational material and teleconsultation of experts independent of their physical position\cite{Wilbur09}.
\end{itemize}

Chapter \ref{sec3_cs} established a service to convert WSIs of various, proprietary formats to DZI, addressing the need to implement multiple image format drivers. But, as stated in \ref{sec4_objective}, a solution to serve proprietary image formats without explicit conversion must be found.

\emph{OpenSlides Python} provides a DZI wrapper. This wrapper can be used to wrap a proprietary WSI and treat it as a DZI\cite{web:openslide}. A DSR can use this to serve a proprietary WSI as DZI to a viewer.

For the reasons mentioned above, the AS will be implemented as a web application. To do so, it will be split into 2 parts: a DSR and a viewer.


\section{Parts of the Annotation Service}
\label{sec4_parts}
As described in section \ref{sec4_methodology}, the AS will be realized in 2 separate parts:
\begin{itemize}
	\item a DSR, called \emph{Annotation Service Server} (ASS)\nmc{ASS}{Annoation Service Server} (see subsection \ref{sec4_assPart})
	\item a viewer, called \emph{Annotation Service Viewer} (ASV)\nmc{ASV}{Annotation Service Viewer} (see subsection \ref{sec4_asvPart})
\end{itemize}

The ASS will be responsible for data management, supplying image data and serving the ASV to the client. The ASV will supply a GUI, including a WSI viewer and all functions specified in section \ref{sec4_functions}.

The two components interact as follows: once the client requested a valid image URL, the ASS will check if the requested WSI is a DZI and, if so, render a ASV with the image path, MPP and file name of the WSI. If the WSI is proprietary, it will be wrapped by OpenSlide. The remaining procedure is then identical to the DZI case.

The ASV is returned to the client and requests the data necessary to view the WSI and it's annotation if they should already exist (this includes configurations, annotations, labels and the image tiles for the current view).

Once loaded, the client can change the current view to maneuver through the different levels and image tiles available, which will be requested by the ASS whenever needed. Additionally, annotations can be made and persisted at any time.

See fig. \ref{fig4_asUml} for a diagram of how the ASS and ASV interact with each other.

\begin{figure}[!h]
	\begin{center}
		\includegraphics[scale=0.4]{img/asUML.png}
		\caption{Activity diagram of ASS and ASV}
		\label{fig4_asUml}
	\end{center}
\end{figure}

\subsection{Annotation Service Server}
\label{sec4_assPart}
As described in section \ref{sec4_methodology}, the ASS serves as a DZR. As such it is responsible for the storage of WSI files and their related metadata\cite{Cornish13}. Additionally, it will be responsible to serve a

First, it serves as a so called \emph{Digital Slide Repository} (DSR). A DSR manages storage of WSIs and their metadata. Additionally, it serves requested image data to a viewer client\cite{Cornish13}, such as the ASV. 

Second, it is responsible for file management. In detail, this means:
\begin{itemize}
	\item persist made annotations in a file
	\item deliver annotation data together with image data
	\item serve list of all available label dictionaries
	\item serve label dictionary entries
	\item save added entries to existing label dictionary
	\item create new, empty label dictionaries
\end{itemize} 

The development of a fully functional web server is not in the scope of this thesis. Therefore, the ASS will run as a local web server. This works around many of the common issues when hosting a web server\cite{web:typicalissues}, such as:

\begin{itemize}
	\item inefficient data or page caching
	\item firewall throughput
	\item internet access throughput
	\item load balance issues
	\item gateway issues
	\item poor security design
	\item connectivity issues
\end{itemize}


\subsection{Annotation Service Viewer}
\label{sec4_asvPart}
The ASV is developed to deploy a GUI through which the pathologist is enabled to view a WSI and annotate it. The ASV is developed in an iterative approach with the help of selected pathologists. After each iteration, the GUI and user experience (UX)\nmc{UX}{User Experience} will be evaluated. This way, the ASV can be adapted to the needs of a real life environment based on the pathologists feedback.

\begin{figure}[H]
	\begin{center}
		\includegraphics[scale=0.2]{img/microdrawUI.png}
		\caption{Microdraw GUI with opened WSI}
		\label{fig4_microdrawUI}
	\end{center}
\end{figure}

The first iteration of the ASV will be based on an open source project called \emph{MicroDraw}\footnote{See \url{https://github.com/r03ert0/microdraw} for more information on the MicroDraw project} (see fig. \ref{fig4_microdrawUI} for MicroDraws GUI).  MicroDraw is a web application to view and annotate \emph{"high resolution histology data"}\cite{web:microdraw2}. The visualization is based on another open source project, called \emph{OpenSeadragon}\cite{web:openseadragon}. Annotations are made possible by the use of \emph{Paper.js}\footnote{See \url{http://paperjs.org/} for more information on Paper.js}. This delivers a baseline for the functionality specified in \ref{sec4_functions} and can be further adjusted to the needs of the ASV.

Apart from the frameworks used, MicroDraw is written in JavaScript using HTML5, CSS3 and jQuery\footnote{See \url{https://jquery.com/} for more information on jQuery}.

\section{Annotation Service Server Implementation}
% uses json to save annotations and dictionaries
% internal structure

The ASS is a local RESTful server written in python (\emph{\textbf{as{\textunderscore}server.py}}). Additional frameworks have been used to improve functionality. Those are:

\begin{itemize}
	\item Flask\cite{web:flask} (see subsection \ref{sec4_flask})
	\item OpenSlide Python\cite{web:openslide} (see subsection \ref{sec4_openslide})
\end{itemize}

The use of the Flask framework makes a certain folder structure necessary\cite{web:flask}. To serve static files, a \emph{"static/"} directory must be present in the base directory of the ASS. The static/ directory contains the CSS, JavaScript, dictionaries and WSIs. To read a WSI with the ASS, it must be placed in static/wsi/[file path/]. The dictionaries can be found in static/dictionaries/, if a manual manipulation of a dictionary (deleting one or deleting/updating an entry in one) becomes necessary.

The ASS can be started from a terminal through the use of a python interpreter:
\begin{lstlisting}
	$ python as_server.py
\end{lstlisting}

Alternatively, python's -m switch can be used:
\begin{lstlisting}
	$ export FLASK_APP=as_server.py
	$ python python -m flask run
\end{lstlisting}

When started without further parameters, the server will listen to the IP 127.0.0.1, port 5000 by default. Another IP address or port can be specified via the -l and -p parameter (see tab. \ref{tab4_assParams} for a complete list of available parameters):
\begin{lstlisting}
	$ python as_server.py -l 192.27.119.89 -p 4711
\end{lstlisting}

\begin{table}[H]
	\begin{center}
		\begin{tabular}{| l | l | r |}
			\hline
			\textbf{parameter} & \textbf{description} & \textbf{default}\\ \hline
			-B, --ignore-bounds & render only the non-empty slide region & false\\ \hline
			-e, --overlap & set overlap between adjacent tiles in pixels & 0 \\ \hline
			-f, --format & set tile format (PNG or JPEG) & JPEG \\ \hline
			-l, --listen & set IP address to listen to & 127.0.0.1\\ \hline
			-p, --port & set port to listen to & 5000\\ \hline
			-Q, --quality & set JPEG compression quality in \% & 100\\ \hline
			-s, --size & set tile size & 256\\ \hline
		\end{tabular}
		\caption{Parameters for as{\textunderscore}server.py}
		\label{tab4_assParams}
	\end{center}
\end{table}

To access a WSI, the URL must be pointed to it, e.g. \url{http://127.0.0.1:5000/wsi/openslide/CMU-1.svs} to access the WSI CMU-1.svs in the directory static/wsi/openslide/.


\subsection{Flask}
\label{sec4_flask}
To give the ASS its server capabilities, Flask was used. Flask considers itself as \emph{"microframework"}, meaning that the development team tries to keep its core \emph{"simple, but extensible"}\cite{web:flask}. It contains a built-in development server, integrated unit testing, RESTful request dispatching and is fairly easy to set up and use. A minimal Flask application can look like this\footnote{The example application code snippet is taken from \cite{web:flask}}:

\begin{lstlisting}[frame=single]
from flask import Flask
app = Flask(__name__)

@app.route('/')
def hello_world():
	return 'Hello, World!'

app.run(host='127.0.0.1', port='5000')
\end{lstlisting}

After the import of the Flask class (line 1), a Web Server Gateway Interface (WSGI)\nmc{WSGI}{Web Server Gateway Interface} object is created\footnote{The WSGI is a standard interface for the communication between web servers and web applications or frameworks in python. The interface has a server and application side. Basically, the server side invokes a callable object that is provided by the application side. The specifics of providing this object are up to the individual server\cite{Brandl16}.} (line 2). The defined \emph{hello\textunderscore}world function (line 5 and 6) is annotated with a \emph{route() decorator} to bind a specific URL to it (line 4). Once the specified URL is requested, Flask knows which function to invoke. Line 8 starts a local server listening on 127.0.0.1:5000.

As mentioned above, a \emph{route() decorator} (see fig. \ref{fig4_routeDecorator}) binds a URL to a function. When bound, the function will be called, once the specified URL is requested by the client\cite{web:flask}.



\begin{figure}[H]
	\begin{center}
		\includegraphics[scale=0.5]{img/route.png}
		\caption{Hello, World! example on how to use Flasks route() decorator (source: \cite{web:flask})}
		\label{fig4_routeDecorator}
	\end{center}
\end{figure}

A bound URL can also contain variable sections, which are marked as \emph{/{\textless}variable name{\textgreater}}. Optionally, a converter can by used to only accept variables of a certain type. This becomes possible by specifying the converter in front of the variable: \emph{/{\textless}converter:variable name{\textgreater}}\cite{web:flask}. See tab. \ref{tab4_converter} for the list of available converters in Flask.

\begin{table}[H]
	\begin{center}
		\begin{tabular}{| l | l |}
			\hline
			\textbf{name} & \textbf{accepted input}\\ \hline
			string & any text without a slash (default)\\ \hline
			int & integer values\\ \hline
			float & floating point values\\ \hline
			path & like string, but also accepts slashes \\ \hline
			any & matches one of the items provided\\ \hline
			uuid & UUID strings\\ \hline
		\end{tabular}
		\caption{Available converters in Flask (source: \cite{web:flask})}
		\label{tab4_converter}
	\end{center}
\end{table}

To bind a URL with one or more variable sections to a function, the corresponding function must have the variable sections as parameters:

\begin{lstlisting}[frame=single]
@app.route('/<slug>_files/<int:level>/<int:col>_
<int:row>.<format>')
def tile(slug, level, col, row, format): ...
\end{lstlisting}

HTTP knows different methods for accessing URLs. By default, a route only answers to GET requests and refuses every other kind with a "405 Method not allowed" http status code. This can be changed by adding the \emph{methods} argument to the route() decorator (see fig. \ref{fig4_methods})\cite{web:flask}.

\begin{figure}[H]
	\begin{center}
		\includegraphics[scale=0.5]{img/HTTPmethods.png}
		\caption{Example use of the method argument (source: \cite{web:flask})}
		\label{fig4_methods}
	\end{center}
\end{figure}

Through the use of decorators a RESTful API was deployed for the ASS (compare subsection \ref{sec4_api}).


\subsection{OpenSlide Python}
\label{sec4_openslide}
To read WSIs, the ASS uses OpenSlide Python, a python interface to the OpenSlide C library. It provides a simple interface for reading WSI. Additionally, it offers a DZI wrapper\cite{web:openslide}, called \emph{DeepZoomGenerator} (DZG)\nmc{DZG}{DeepZoomGenerator}, which can be used to create Deep Zoom tiles on demand. WSIs of the following formats are supported:

\begin{itemize}
	\item BIF
	\item NDPI
	\item MRXS
	\item SCN
	\item SVS
	\item SVSLIDE
	\item TIF
	\item TIFF
	\item VMS
	\item VMU
\end{itemize}

Through OpenSlide, the ASS can read a proprietary WSI as a so called \emph{OpenSlide} object (see line 4). As such it has methods to access available metadata, image tiles, the thumbnail and associated images. This OpenSlide object can be wrapped with a DZG to enable DZI support\cite{web:openslide}. To do so, the OpenSlide object needs to be passed into the constructor of the DZG (see line 5), together with a number of optional parameters (see tab \ref{tab4_DZGparam} for parameters and their default values):

\begin{lstlisting}[frame=single]
from openslide import open_slide
from openslide.deepzoom import DeepZoomGenerator

slide = open_slide(slide path)
dzg = DeepZoomGenerator(slide[, tile_size, overlap, limit_bounds])
\end{lstlisting}

\begin{table}[H]
	\begin{center}
		\begin{tabular}{| p{2.5cm} | p{2cm} | p{5.5cm} |}
			\hline
			\textbf{parameter} & \textbf{type} & \textbf{description}\\ \hline
			osr & OpenSlide, ImageSlide & the slide object
			\\ \hline
			tile{\textunderscore}size & integer & the width and height of a single tile (254)\\ \hline
			overlap & integer & the number of extra pixels to add to each interior edge of a tile (1)\\ \hline
			limit{\textunderscore}bounds & boolean & true to render only the non-empty slide region (false)\\ \hline
		\end{tabular}
		\caption{DeepZoomGenerator parameters (with default values, source: \cite{web:openslide})}
		\label{tab4_DZGparam}
	\end{center}
\end{table}

Once created, the DZG can give numerous informations about the tiles and levels of the wrapped WSI\footnote{See \cite{web:openslide} for an in-depth list of functions}. Of special importance are the \emph{get{\textunderscore}dzi} and \emph{get{\textunderscore}tile} functions. The get{\textunderscore}file(format) function creates a string, containing the metadata of the .dzi-file\footnote{See subsection \ref{sec2_openFormats} - Deep Zoom Images}. The \emph{format} parameter specifies the format of the individual tiles (PNG or JPEG). The get{\textunderscore}tile(level, address) function returns an image of the tile corresponding to the supplied parameter values (see tab. \ref{tab4_getTileParams}). The returned tile is either PNG or JPEG, depending on the value on the value passed to the format parameter of the get{\textunderscore}file function.

\begin{table}[H]
	\begin{center}
		\begin{tabular}{| p{1.5cm} | p{1.5cm} | p{7cm} |}
			\hline
			\textbf{name} & \textbf{type} & \textbf{description}\\ \hline
			level & integer & the DZI level to get the tile from \\ \hline
			address & tuple & the address of the thile within the level as a (column, row) tuple\\ \hline
		\end{tabular}
		\caption{DeepZoomGenerators get{\textunderscore}tile parameters (source: \cite{web:openslide})}
		\label{tab4_getTileParams}
	\end{center}
\end{table}

Through the use of those 2 functions, the ASS can create the metadata of a DZI on the fly and pass it to the ASV. The ASV then requests the individual tiles needed for the current view in return, which are generated by the DZG from the original WSI on demand.


\subsection{Annotation Service Server RESTful API}
\label{sec4_api}
To communicate with the ASS a RESTful API was deployed. This is realized with Flasks route() decorators\footnote{See subsection \ref{sec4_flask}}. The listing below gives an overview over the URLs that the ASS RESTful API offers (in the style of \textbf{URL} (method): \emph{function(parameters)}), followed by a brief description of the functionality\footnote{See appendix \ref{sec_B1} for a detailed documentation of the ASS functions}:

\begin{enumerate}[(1) -]
	\item\textbf{/wsi/{\textless}path:file{\textunderscore}path{\textgreater}.dzi} (GET): \emph{index{\textunderscore}dzi(file{\textunderscore}path)}\\
	This URL is used to request a viewer with a specified DZI from the ASS. The ASS then renders an ASV and passes the DZIs file path, its microns per pixel (MPP)\nmc{MPP}{Microns per Pixel} and its name to it. The ASV feeds the path to OpenSeadragon, which then views the DZI. The MPP are used to calculate the actual image size in $\mu$m for the scale. Lastly, the file name is used to change the name of the browser tab accordingly.
	
	\item \textbf{/wsi/{\textless}path:file{\textunderscore}path{\textgreater}} (GET): \emph{index{\textunderscore}wsi(file{\textunderscore}path)}\\
	Works similar to (1), except that the requested image is not of the DZI format. To view it in OpenSeadragon anyway, an instance of the DZG is created which wraps the proprietary WSI. Then, a specific path ("\emph{/slide.dzi}") is handed to the ASV.
	
	\item \textbf{/{\textless}slug{\textgreater}.dzi} (GET): \emph{dzi(slug)}\\
	Once the ASV requests slide.dzi, the DZG builds a response with the descriptive DZI file, created from the proprietary WSI format (via its \emph{.get{\textunderscore}dzi(format)} function, with \emph{format} being the file format of the tiles) and serves it to the ASV.
	
	\item \textbf{/{\textless}slug{\textgreater}{\textunderscore}files/{\textless}int:level{\textgreater}{\textunderscore}{\textless}int:col{\textgreater}{\textunderscore}{\textless}int:row{\textgreater}.{\textless}format{\textgreater}}\\(GET): \emph{tile(slug, level, col, row, format)}\\
	If an original DZI is requested, there is a \emph{/wsi/} in front of the URL. Therefore, this URL only triggers if (3) was called before. This way OpenSeadragon requests the separate tiles needed to fill the current view of the user. This is done via the DZG \emph{.get{\textunderscore}tile(level, address)} function. \emph{Level} describes the requested level, while \emph{address} is a tuple with the x (col) and y (row) position of the requested tile.
	
	\item \textbf{/saveJson} (POST): \emph{saveJson()}\\
	This URL is used when the user wants to save made annotations. The name of the JSON file and the content to write into it will be send via the POST request. The content of the POST request can be accessed via Flasks \emph{Request} object\cite{web:flask} in the following fashion:
	\begin{lstlisting}[frame=single]
	post_data = request.form
	source = post_data.get('file', default='')
	content = post_data.get('content', default='{}').
	encode('utf-8')
	\end{lstlisting}
	
	\item \textbf{/loadJson} (GET): \emph{loadJson()}\\
	(6) is used to load a JSON file, may that be the \emph{configuration.json}, a dictionary or saved annotations. The name of the source is passed as parameter (?src=[file]) to the ASS. Similar to (5), it can be accessed with the Request object\cite{web:flask}: \emph{request.args.get(parameter, default value)}.
	
	\item \textbf{/createDictionary} (GET): \emph{createDictionary()}\\
	If the user sends the command to create a new dictionary, this URL is called. The ASS then creates a new, empty dictionary file. The name of the dictionary is passed as a URL parameter (?name=[name]) and then acquired in the same fashion as in (5). The ASS also opens the configuration file and changes the currently selected dictionary to the newly created one.
	
	\item \textbf{/getDictionaries} (GET): \emph{getDictionaries()}\\
	When called, the ASS looks up the content of its \emph{dictionaries} folder and returns a list with the found file names or -1 in the case of an error.
\end{enumerate}


% python web server, flask, openslide, java script, html5, css, jquery
\section{Annotation Service Viewer Implementation}
% paper
% osd
% jquery
% functions?
% loading process
% inizialization of paper and openseadragon
