\chapter{Introduction}

\section{Motivation}


\section{Research Objective}
The objective of this thesis is the conceptualization and implementation of tools for whole slide images\footnote{See chapter 2.1.x}\nmc{WHI}{Whole Slide Image}, which allow for their annotation and a further usage in neural networks\footnote{See chapter 2.1.x}\nmc{NN}{Neural Networks}.
As a requirement, the tools have to be implemented in the form of microservices\footnote{See chapter 2.1.x}, each one with their own short documentation, including instructions for installation, usage and some examplary use cases. To achieve this goal, the implementation of 3 microservices is necessary.

The first microservice needs to be capable of converting a given set of image formats into the so called \emph{Deep Zoom Image Format}\footnote{See chapter 2.1.x}. The supported image formats are \emph{.bif, .mrxs, .ndpi, .scn, .svs, .svslide, .tif, .tiff, .vms} and \emph{.vmu}, in accordance with the capabilities of the Openslide framework\cite{Goode13}.

\section{About this thesis}
Apart from the \emph{Introduction}, there are 5 more chapters in this thesis.

\emph{Chapter 2 - Background} defines some terminoligy and the general, required process chain which are all necessary to understand further chapters of this thesis. Furthermore, 3 microservices will be introduced in short.

\emph{Chapter 3 - Methodology} gives an overview over the current state of research for each microservice, as well as best practices. 

\emph{Chapter 4 - Implementation} goes into further details about how each microservice is implemented and which software and frameworks were used for that.

\emph{Chapter 5 - Discussion} will introduce a measurement for each microservice to measure its success. It will discuss the test setup as well as list the results.

\emph{Chapter 6 - Conclusion} will interpret the Results from Chapter 5 and analyze them closer. Furthermore, it will give an idea of what steps are to be taken next in the future.