\documentclass{report}

\usepackage[utf8]{inputenc}
\usepackage{amsmath}
\usepackage{cite}
\usepackage[dvipdfmx]{graphicx}
\usepackage{bmpsize}
\usepackage[hyphens]{url}
\usepackage{hyperref}
\usepackage{enumerate}
\usepackage{morefloats}
\usepackage[nottoc]{tocbibind}
\usepackage[english]{babel}
\usepackage{float}
\usepackage{listings}
\usepackage{color}
\usepackage[toc,page]{appendix}
\usepackage{array}
\usepackage{longtable}
\usepackage[htt]{hyphenat}
\usepackage{amsfonts}
\usepackage{subcaption}

% --- Abk�rzungsverzeichnis: ----------------------------
% START % N�heres siehe http://my.opera.com/timomeinen/blog/show.dml/68644
\usepackage[intoc]{nomencl}
% Befehl umbenennen in abk
\let\nmc\nomenclature
% Punkte zw. Abk�rzung und Erkl�rung
\setlength{\nomlabelwidth}{.20\hsize}
\renewcommand{\nomlabel}[1]{#1 \dotfill}
% Zeilenabst�nde verkleinern
\setlength{\nomitemsep}{-\parsep}
\makenomenclature
%--------------------------------------------------------

\definecolor{codegreen}{rgb}{0,0.6,0}
\definecolor{codegray}{rgb}{0.5,0.5,0.5}
\definecolor{codepurple}{rgb}{0.58,0,0.82}
\definecolor{backcolour}{rgb}{0.95,0.95,0.92}

\newcommand{\review}[1]{
	\textbf{\textcolor{red}{#1}}
}

\newcolumntype{R}[1]{>{\raggedleft\let\newline\\\arraybackslash\hspace{0pt}}m{#1}}

\lstdefinestyle{mystyle}{
    backgroundcolor=\color{backcolour},   
    commentstyle=\color{codegreen},
    keywordstyle=\color{magenta},
    numberstyle=\tiny\color{codegray},
    stringstyle=\color{codepurple},
    basicstyle=\footnotesize,
    breakatwhitespace=false,         
    breaklines=true,                 
    captionpos=b,                    
    keepspaces=true,                 
    numbers=left,                    
    numbersep=5pt,                  
    showspaces=false,                
    showstringspaces=false,
    showtabs=false,                  
    tabsize=2
}
 
\lstset{style=mystyle}

% js support for lstlisting
\definecolor{lightgray}{rgb}{.9,.9,.9}
\definecolor{darkgray}{rgb}{.4,.4,.4}
\definecolor{purple}{rgb}{0.65, 0.12, 0.82}

\lstdefinelanguage{JavaScript}{
	keywords={typeof, new, true, false, catch, function, return, null, catch, switch, var, if, in, while, do, else, case, break},
	keywordstyle=\color{blue}\bfseries,
	ndkeywords={class, export, boolean, throw, implements, import, this},
	ndkeywordstyle=\color{darkgray}\bfseries,
	identifierstyle=\color{black},
	sensitive=false,
	comment=[l]{//},
	morecomment=[s]{/*}{*/},
	commentstyle=\color{purple}\ttfamily,
	stringstyle=\color{red}\ttfamily,
	morestring=[b]',
	morestring=[b]"
}

\interfootnotelinepenalty=10000

\begin{document}

\begin{titlepage}
	\centering
	{\huge\textbf{Development of a guided tagging tool for Whole Slide Images\\}}
	\vspace{1cm}
	{\huge Master Thesis\\}
	\vspace{2cm}
	{\large Submitted in partial fulfillment of the requirements for the degree
	of\\}
	\vspace{0.5cm}
	{\large \textbf{Master of Science (M.Sc.)}\\ 
	in Applied Computer Science\\}
	\vspace{0.5cm}
	{\large at the\\}
	\vspace{0.5cm}
	{\large Berlin University of Applied Sciences (HTW)\\}
	\vspace{0.5cm}
	\includegraphics[scale=1]{img/HTW_Logo_quer_rgb.jpg}
	\\
	\vspace{1cm}
	\raggedright
	{\large First Supervisor: Prof. Dr. Peter Hufnagl\\}
	\vspace{0.25cm}
	{\large Second Supervisor: Diplom Informatiker Benjaming Voigt\\}
	\centering
	\vspace{1.5cm}
	{\large Submitted by: \\}
	\vspace{0.25cm}
	{\large \textbf{Sascha Nawrot (B.Sc.)}}
	\vfill
		{\large Berlin, \today{}\par}
\end{titlepage}

\newpage
\chapter*{Preface}
Hello, this is the preface

\newpage
\begin{abstract}
The digitalization of pathologic tissue samples (so called whole slide images) enables the use of image processing and analysis techniques to support pathology experts in their work. The area of digital pathology can greatly benefit from introducing neural networks. Neural networks have proven to be ideally suited for image classification and pattern recognition tasks. In order to be used efficiently, they need training. To do so, training samples and ground truth are needed. Unfortunately, context specific training samples are hard to come by in this medical context, thus the needed samples must be created. One way to create training samples are image annotations. 

The lack of an official standard resulted in a number of proprietary solutions and image formats by different vendors, including viewers and annotation tools. Those tools only work on specific platforms and formats. Additionally, most of those tools are not specialized in annotating images and include many other functions, making them difficult to use. This makes the annotation of whole slide images difficult and cumbersome.

Therefore, the goal of this thesis is to introduce novel, lightweight open source annotation tools for whole slide images that enable deep learning and pathology experts to cooperate in creating training samples by annotating regions of interest in whole slide images, regardless of platform and format, in a fast and easy manner. Additionally, the made annotations will be extracted into regular image (JPEG, PNG) and text (TXT) files, to ensure easy-to-use and non-proprietary training samples.
\end{abstract}
\newpage

\tableofcontents
\newpage

\chapter{Introduction}

\section{Motivation}
The medical discipline of pathology is in a digital transformation. Instead of looking at tissue samples through the means of traditional light microscopy, it is now possible to digitize those samples. This digitalization is done with the help of a so called slide scanner. The result of such an operation is a \emph{whole slide image} (WSI)\nmc{WSI}{Whole Slide Image}\cite{Cornish13}. The digital nature of WSIs opens the door to the realm of image processesing and analysis which yields certain benefits, such as the use of image segmentation and registration methods to support the pathologist in his/her work.

A very promising approach to image analysis is the use of deep learning, also known as \emph{neural networks} (NN)\nmc{NN}{Neural Network}. These are a group of computational models inspired by our current understanding of biological NN. The construct of many interconnected neurons is considered a NN (both in the biological and artificial context). Each single one of those neurons has input values and an output value. Once the input reaches a certain trigger point, the cell in the neuron sends a signal as output. The connections between the neurons are weighted and can dampen or strengthen a signal. Because of this, old pathways can be blocked and new ones created. In other words, a NN is capable of "learning"\cite{Kriesel07}. This is a huge advantage compared to other software models. While certain problems are "easier" to solve in a sequential, algorithmic fashion (say an equation or the towers of hanoi), certain problems (e.g. image segmentation or object recognition) are very complex, so that new approaches are needed, while other problems can not be solved algorithmically at all. With the use of adequate training samples, a NN can learn to solve a problem, much like a human.

\begin{figure}[ht]
	\begin{center}
		\includegraphics[scale=0.3]{img/deepVisual.png}
		\caption{Example results of the in \cite{Karpathy15} introduced model (source: \url{http://cs.stanford.edu/people/karpathy/deepimagesent/})}
		\label{fig1_karpathy}
	\end{center}
\end{figure}

In the recent past the use of NN enabled major breakthroughs, especially in the area of image classification and object recognition. Karpathy and Fei-Fei, for example, created a NN that is capable of describing an image or a scene using natural language text blocks \cite{Karpathy15} (see fig. \ref{fig1_karpathy} for a selection of examples).

There is enormous potential in the use of NN in the digital pathology as well, but to transfer these models and technologies, certain obstacles must be overcome. One of those is the need for proper training samples. While generally there are large amounts of WSIs (e.g. publicly available at the Cancer Genome Atlas\footnote{\url{https://gdc-portal.nci.nih.gov/}}), most of them will not be usable as a training sample without further preparation.

A possible way to prepare them is by using image annotation: tagging regions of interest (ROI)\nmc{ROI}{Region of Interest} on an image and assigning labels or keywords as metadata to those ROIs. These can be added to the WSIs, stored and later used for training. The result of such an approach could be similar to the one of Karpathy and Fei-Fei\cite{Karpathy15}, but with a medical context instead of daily situations.


\section{Research Objective}
\label{sec1_researchObjective}
The goal of this thesis is the conceptualization and implementation of tools to enable deep learning and pathology experts to cooperate in annotating WSIs to create a ground truth for the later use in NNs. In order to do so, it must be possible to open a given WSI with a viewer, add annotations to it and persist those annotations. Additionally, persisted annotations must be extracted from the WSI to be used as ground truth.

%Since there is no WSI file format standard, vendors developed their own proprietary solutions\cite{Cornish13}. This either leads to


%\begin{enumerate}[(i)]
%	\item locking-in on a specific vendor or
%	\item separate handling of each proprietary format
%\end{enumerate}

%(i) would render the whole process vendor specific, limiting its use drastically. (ii) would not render the process chain vendor specific but demand additional work and maintenance, due to the need of implementing multiple drivers for different formats. To counteract this, open file formats, such as the Deep Zoom Image format, have been specified\cite{Cornish13}. Additionally, converters and wrappers exist, that help to void the stated issues (i) and (ii). 

This thesis has three objectives:
\begin{enumerate}[(1)]
	\item There is no standard for WSI files, therefore vendors developed their own proprietary solutions\cite{Cornish13}. This either leads to a vendor lock-in or separate handling of each proprietary format. To avoid both cases, a conversion tool that turns proprietary WSI formats into an open format must be introduced.
	\item The deployment of a WSI viewer tool. This WSI viewer must be capable of adding annotations to a WSI and persisting them. As stated earlier, the tool is intended to be used by deep learning and pathology experts. An intuitive and easy-to-understand graphical user interface (GUI)\nmc{GUI}{Graphical User Interface} is necessary to avoid long learning periods and create willingness to actually use the tool.
	\item The implementation of a tool that is capable of turning persisted annotations into a format usable as ground truth.
\end{enumerate}

It is explicitly stated, that the intention of this thesis is to introduce tools that are used by deep learning and pathology experts to create a ground truth for NN. The intention is \textbf{not} to create a tool for analyzing and diagnosing WSIs, that is capable of competing with existing industry solutions.


\section{About this thesis}
This thesis contains 6 chapters.

\emph{Chapter 1 - Introduction} and \emph{2 - Background} address the scope, background and vocabulary of this thesis.

The chapters 3 to 5 address the components described in the last section: \emph{chapter 3 - Conversion Service} will describe a tool for image conversion, \emph{chapter 4 - Annotation Service} will describe a tool for image annotation and \emph{chapter 5 - Tessellation Service} will describe an extraction tool, to prepare the annotations made with the Annotation Service for the use in a NN.

Finally, \emph{Chapter 6 - Conclusion} will discuss and conclude the findings of the aforementioned chapters.
\chapter{Background}
\section{Image Formats}
not only for the purpose of unification, but also to add the deep zoom feature\footnote{See chap. 2.1.1} to the images (see fig. 2.3). This is of special importance, since an average WSI with 1,600 megapixels has a size of approximately 4.6 GB\cite{Farahanil15}.


\subsection{Open WSI Formats}
\subsubsection{Deep Zoom Images}

\begin{figure}[H]
	\begin{center}
		\includegraphics[scale=0.5]{img/dzi_pyramid.png}
		\caption{3 consecutive levels of a dzi image (source: https://i-msdn.sec.s-msft.com/dynimg/IC141135.png)}
		\label{fig:fig2.1}
	\end{center}
\end{figure}

The Deep Zoom Image Format (dzi)\nomenclature{dzi}{Deep Zoom Image Format} is an xml-based file format maintained by Microsoft to improve performance and quality in the handling of large image files. For this purpose an image is represented in a tiled pyramid (see fig. 2.1).

As seen in fig. 2.1 there are multiple versions of a single image in different resolutions. Each resolution in the pyramid is called a \emph{level}. At each level the image is scaled down by the factor 4 (2 in each dimension). Furthermore, the image gets tiled up into $256^2$ tiles (256 in each dimension)\cite{web:dzi}.

If a viewer wants to view a certain area of the image (e.g. the highlighted tile in the last image in fig. 2.1), only the corresponding tiles need to be loaded. This saves large amounts of bandwidth and memory. The same goes for a viewer, who is zoomed out very far. In such a view the full level of detail isn't needed, so that a version from a lower level can be loaded.

A dzi file consists of two parts: a describing .xml file\footnote{Frameworks like \emph{OpenSeaDragon} also support further formats, such as .json.} and a folder with more subfolder. Each subfolder describes a level and as such contains all the tiles for that particular level.


\section{Short Introduction to Neural Networks}
The objective of this thesis ultimately is to create training samples for NN\footnote{see chap. 1.2}. Before going into other details, it is necessary to clarify what NN are, how they work, why they need training samples and what they use them for\footnote{It should be mentioned that the scope of NN is huge and worth a whole thesis by themselves. This is nothing more than a short introduction and further consultation of literature (e.g. \cite{Stergiou96},\cite{Bourg04},\cite{Egmont-Petersen02},\cite{Kriesel07},\cite{Shiffman12}) is highly recommended.}.

Artificial Neural Networks (NN)\nmc{NN}{Neural Networks} are a group of models inspired by Biological Neural Networks (BNN) \nmc{BNN}{Biological Neural Networks}. BNNs can be described as an interconnected web of neurons (see fig 2.x), whose purpose it is to transmit information in the form of electrical signals. A neuron receives input via dendrites and send output via axons\cite{Shiffman12}. An average human adult brain contains about $10^{11}$ neurons. Each of those receives input from about $10^4$ other neurons. If their combined input is strong enough, the receiving neuron will send an output signal to other neurons\cite{Bourg04}.

\begin{figure}[H]
	\begin{center}
		\includegraphics[scale=0.7]{img/bnn.png}
		\caption{Neuron in a BNN (source:\cite{Shiffman12})}
		\label{fig:fig2.2}
	\end{center}
\end{figure}

NN are much simpler in comparison\footnote{Usually, they don't have much more than a few dozen neurons\cite{Bourg04}.}, but generally work in the same fashion.

One of the biggest strengths of a NN, much like a BNN, is the ability to adapt by learning\footnote{As humans, NN learn by training\cite{Shiffman12}.}. This adaption is based on \emph{weights} that are assigned to the connections between single neurons. Fig 2.x shows an exemplary NN with neurons and the connections between them.

\begin{figure}[H]
	\begin{center}
		\includegraphics[scale=1.0]{img/NN.png}
		\caption{Exemplary NN (source:\cite{Shiffman12})}
		\label{fig:fig2.3}
	\end{center}
\end{figure}

Each line in fig, 2.x represents a connection between 2 neurons. Those connections are a one-directional flow of information, each assigned with a specific weight. This weight is a simple number that is multiplied with the incoming/outgoing signal and therefore weakens or enhances it. They are the defining factor of the behavior of a NN. Determining those values is the purpose of training a NN\cite{Bourg04}.

According to \cite{Shiffman12}, some of the standard use cases for NN are:

\begin{itemize}
	\item Pattern Recognition
	\item Time Series Prediction
	\item Signal Processing Perceptron
	\item Control
	\item Soft Sensors
	\item Anomaly Detection
\end{itemize}


\subsection{Methods of Learning}
There are 3 general strategies when it comes to the training of a NN\cite{Bourg04}. Those are:

\begin{enumerate}
	\item Supervised Learning
	\item Unsupervised Learning
	\item Reinforcement Learning (a variant of Unsupervised Learning\cite{Rojas96})
\end{enumerate}

\emph{Supervised Learning} is a strategy that involves a training set to which the correct output is known, as well as an observing teacher. The NN is provided with the training data and computes its output. This output is compared to the expected output and the difference is measured. According to the error made, the weights of the NN are corrected. The magnitude of the correction is determined by the used learning algorithm\cite{Rojas96}.

\emph{Unsupervised Learning} is a strategy that is required when the correct output is unknown and no teacher is available. Because of this, he NN must organize itself\cite{Shiffman12}. \cite{Rojas96} makes a distinction between 2 different classes of unsupervised learning:

\begin{itemize}
	\item reinforced learning
	\item competitive learning
\end{itemize}

Reinforced learning adjusts the weights in such a way, that desired output is reproduced. For example, a robot in a maze. If the robot can drive straight without any hindrances, it can associate this sensory input with driving straight (desired outcome). As soon as it approaches a turn, the robot will hit a wall (non-desired outcome). To prevent it from hitting the wall it must turn, therefore the weights of turning must be adjusted to the sensory input of being at a turn. Another example is \emph{Hebbian learning}\footnote{see \cite{Rojas96} for further information}\cite{Rojas96}.

In competitive learning, the single neurons compete against each other for the right to give a certain output for an associated input. Only one element in the NN is allowed to answer, so that other, competing neurons are inhibited\cite{Rojas96}.


\subsection{The Perceptron}
The perceptron was invented by Rosenblatt at the Cornell Aeronautical Laboratory in 1957\cite{Rosenblatt58}. It is the computational model of a single neuron and as such, the simplest NN possible\cite{Shiffman12}. A perceptron consists of one or more inputs, a processor and a single output (see fig. 2.x)\cite{Rosenblatt58}.

\begin{figure}[H]
	\begin{center}
		\includegraphics[scale=0.7]{img/perceptron.png}
		\caption{Perceptron by Rosenblatt (source:\cite{Shiffman12})}
		\label{fig:fig2.5}
	\end{center}
\end{figure}

This can be directly compared to the neuron in fig. 2.x, where:
\begin{itemize}
	\item input = dendrites
	\item processor = cell
	\item output = axon
\end{itemize}

A perceptron is only capable of solving \emph{linearly separable} problems, such as logical \emph{AND} and \emph{OR} problems. To solve non-linearly separable problems, more then one perceptron is required\cite{Rosenblatt58}. Simply put, a problem is linearly separable, if it can be solved with a straight line (see fig. 2.x), otherwise it is considered a non-linearly separable problem (see fig. 2.x).

\begin{figure}[H]
	\begin{center}
		\includegraphics[scale=0.6]{img/lsp.png}
		\caption{Examples for linearly separable problems (source:\cite{Shiffman12})}
		\label{fig:fig2.7}
	\end{center}
\end{figure}

\begin{figure}[H]
	\begin{center}
		\includegraphics[scale=0.75]{img/nlsp.png}
		\caption{Examples for non-linearly separable problems (source:\cite{Shiffman12})}
		\label{fig:fig2.8}
	\end{center}
\end{figure}


\subsection{Multi-layered Neural Networks}
To solve more complex problems, multiple perceptrons can be connected to from a more powerful NN. A single perceptron might not be able to solve \emph{XOR}, but one perceptron can solve \emph{OR}, while the other can solve \emph{$\neg$AND}. Those two perceptrons combined can solve \emph{XOR}\cite{Shiffman12}.

If multiple perceptrons get combined, they create layers. Those layers can be separated into 3 distinct types\cite{Stergiou96}:
\begin{itemize}
	\item input layer
	\item hidden layer
	\item output layer
\end{itemize}

A typical NN will have an input layer, which is connected to a number of hidden layers, which either connect to more hidden layers or, eventually, an output layer (see fig. 2.x for a NN with one hidden layer).

\begin{figure}[H]
	\begin{center}
		\includegraphics[scale=0.8]{img/mlp.png}
		\caption{NN with multiple layers (source: http://docs.opencv.org/2.4/\textunderscore images/mlp.png)}
		\label{fig:fig2.2}
	\end{center}
\end{figure}

As the name suggests, the input layer gets provided with the raw information input. Depending on the internal weights and connections inside the hidden layer, a representation of the input information gets formed. At last, the output layer generates output, again based on the connections and weights of the hidden and output layer\cite{Stergiou96}.

Training this kind of NN is much more complicated than training a simple perceptron, since weights are scattered all over the NN and its layers. A solution to this problem is called \emph{backpropagation}\cite{Shiffman12}.


\subsubsection{Backpropagation}
Training is an optimization process. To optimize something, a metric to measure has to be established. In the case of backpropagation, this metric is the accumulated output error of the NN to a given input\footnote{To do so, it is necessary to know the right answer. Therefore, backpropagation is part of the supervised learning process.}. There are several ways to calculate this error, with the \emph{mean square error}\footnote{Mean square error is the average of the square of the differences of two variables, in this case the expected and the actual output.} being the most common one\cite{Bourg04}.

Finding the optimal weights is an iterative process of the following steps:
\begin{enumerate}
	\item start with training set of data with known output
	\item initialize weights in NN
	\item for each set of input, feed the NN and compute the output
	\item compare calculated with known output
	\item adjust weights to reduce error
\end{enumerate}

There are 2 possibilities in how to proceed. The first one is to compare results and adjust weights after each input/output-cycle. The second one is to calculate the accumulated error over a whole iteration of the input/output-cycle. Each of those iterations is known as an \emph{epoch}\cite{Bourg04}.


\section{Microservices}
The following section elaborates on the concept of \emph{Microservices} (MS)\nmc{MS}{Microservice}, defining what they are, listing their pros and cons, as well as explaining why this approach was chosen over a monolithic approach. A monolithic software solution is described by \cite{Lewis14} as follows:
\begin{quotation}
	"[...] a monolithic application [is] built as a single unit. Enterprise Applications are often built in three main parts: a client-side user interface (consisting of HTML pages and javascript running in a browser on the user's machine) a database (consisting of many tables inserted into a common, and usually relational, database management system), and a server-side application. The server-side application will handle HTTP requests, execute domain logic, retrieve and update data from the database, and select and populate HTML views to be sent to the browser. This server-side application is a monolith - a single logical executable. Any changes to the system involve building and deploying a new version of the server-side application."
\end{quotation}


\subsection{Definition}
MS are an interpretation of the Service Oriented Architecture. The concept is to separate one monolithic software construct into several smaller, modular pieces of software\cite{Wolff16}. As such, MS are a modularization concept. However, they differ from other such concepts, since MS are independent from each other. This is a trait, other modularization concepts usually lack\cite{Wolff16}. As a result, changes in one MS don't bring up the necessity of deploying the whole product cycle again, but just the one service. This can be achieved by turning each MS into an independent process with its own runtime\cite{Lewis14}.

This modularization creates an information barrier between different MS. Therefore, if MS need to share data or communicate with each other, light weight communication mechanisms must be established, such as a RESTful API\cite{Riggins15}.

Even though MS are more a concept than a specific architectural style, certain traits are usually shared between them\cite{Riggins15}. According to \cite{Riggins15} and \cite{Lewis14}, those are:

\begin{enumerate}[(a)]
	\item \textbf{Componentization as a Service:} bringing chosen components (e.g. external libraries) together to make a customized service
	\item \textbf{Organized Around Business Capabilities:} cross-functional teams, including the full range of skills required to achieve the MS goal
	\item \textbf{Products instead of Projects:} teams own a product over its full lifetime, not just for the remainder of a project
	\item \textbf{Smart Endpoints and Dumb Pipes:} each microservice is as decoupled as possible with its own domain logic
	\item \textbf{Decentralized Governance:} enabling developer choice to build on preferred languages for each component.
	\item \textbf{Decentralized Data Management:} having each microservice label and handle data differently
	\item \textbf{Infrastructure Automation:} including automated deployment up the pipeline
	\item \textbf{Design for Failure:} a consequence of using services as components, is that applications need to be designed so that they can tolerate the failure of single or multiple services
\end{enumerate}

Furthermore, \cite{Bugwadia15} defined 5 architectural constraints, which should help to develop a MS:

\begin{enumerate}[(1.)]
	\item \textbf{Elastic}\\
	The elasticity constraint describes the ability of a MS to scale up or down, without affecting the rest of the system. This can be realized in different ways. \cite{Bugwadia15} suggests to architect the system in such a fashion, that multiple stateless instances of each microservice can run, together with a mechanism for Service naming, registration, and discovery along with routing and load-balancing of requests.
	\item \textbf{Resilient}\\
	This constraint is referring to the before mentioned trait (h) - \emph{Design for Failure}. The failure of or an error in the execution of a MS must not impact other services in the system.
	\item \textbf{Composable}\\
	To avoid confusion, different MS in a system should have the same way of identifying, representing, and manipulating resources, describing the API schema and supported API operations.
	\item \textbf{Minimal}\\
	A MS should only perform one single business function, in which only semantically closely related components are needed.
	\item \textbf{Complete}\\
	A MS must offer a complete functionality, with minimal dependencies to other services. Without this constraint, services would be interconnected again, making it impossible to upgrade or scale individual services.
\end{enumerate}


\subsection{Advantages and Disadvantages}
% pros
One big advantage of this modularization is that each service can be written in a different programming language, using different frameworks and tools. Furthermore, each microservice can bring along its own support services and data storages. It is imperative for the concept of modularization, that each microservice has its own storage of which it is in charge of\cite{Wolff16}.

The small and focused nature of MS makes scaling, updates, general changes and the deploying process easier. Furthermore, smaller teams can work on smaller code bases, making the distribution of know how easier\cite{Riggins15}.

Another advantage is how well MS plays into the hands of agile, scrum and continuous software development processes, due to their previously discussed inherent traits.

% cons
The modularization of MS doesn't only yield advantages. Since each MS has its own, closed off data management\footnote{See 2.3.1(f) (\emph{Decentralized Data Management})}, interprocess communication becomes a necessity. This can lead to communicational overhead which has a negative impact on the overall performance of the system\cite{Wolff16}.

2.3.1(e) (\emph{Decentralized Governance}) can lead to compatibility issues, if different developer teams chose to use different technologies. Thus, more communication and social compatibility between teams is required. This can lead to an unstable system which makes the deployment of extensive workarounds necessary\cite{Riggins15}.

It often makes sense to share code inside a system to not replicate functionality which is already there and therefore increase the maintenance burden. The independent nature of MS can make that very difficult, since shared libraries must be build carefully and with the fact in mind, that different MS may use different technologies, possibly creating dependency conflicts.


\subsection{Conclusion}	
After consideration of the advantages and disadvantages of MS, the author decided in favor of using them. This is mainly due to the fact of working alone on the project, negating some of their inherent disadvantages:
\begin{itemize}
	\item Interprocess communication doesn't arise between the single stages of the process chain, since they have a set order\footnote{E.g. it wouldn't make sense trying extract a training sample without converting or annotating a WSI first.}
	\item Different technologies may be chosen for the single steps of the process chain, however, working alone on the project makes technological incompatibilities instantly visible
	\item The services shouldn't share functionality, therefore there should be no need for shared libraries
\end{itemize}
This makes the advantages outweight the disadvantages clearly:
\begin{itemize}
	\item different languages and technologies can be used for every single step of the process chain, making the choice of the most fitting tool possible
	\item WSIs take a heavy toll on memory and disk space due to their size; the use of MS allows each step of the chain to handle those issues in the most suitable way for each given step
	\item separating the steps of the process chain into multiple MS makes for a smaller and easier maintainable code base
	\item other bachelor/master students may continue to use or work on this project in the future, making the benefit of a small, easily maintainable code base twice as important
	\item the implementation of the project will happen in an iterative, continuous manner, which is easily doable with the use of MS
\end{itemize}


\section{Process Chain}
This section and its following subsections are dedicated to establish the process chain necessary to accomplish the research objectives stated in 1.2(a) - 1.2(c). The usual procedure would look as follows:

\begin{enumerate}[(1.)]
	\item convert chosen WSI $img^{wsi}_i$ to DZI format $img^{dzi}_i$
	\item open $img^{dzi}_i$ in a viewer $V$
	\item annotate $img^{dzi}_i$ in $V$
	\item persist annotations $A_i$ on $img^{dzi}_i$ in a file $f_{(A_i)}$
	\item create training sample $ts_i$ by extracting the information of $A_i$ in correspondence to $img^{dzi}_i$
\end{enumerate}

While it only makes sense to run (1.) once per $img^{wsi}_i$ to create $img^{dzi}_i$, steps (2.) - (4.) can be repeated multiple times, so that there is no need to finish the annotation of an image in one session. That makes it necessary to not only save but also load annotations. Therefore, the loading of already made annotations can be added as step (2.5). This also enables the user of editing and deleting already made annotations. Because of this, step (5.) also needs to be repeatable.

The single steps of the process chain will be sorted into semantic groups. Each group will be realized by its own MS, as stated in 2.3. The semantic groups are: conversion (1.), extraction (5.) and viewing and annotation (2. - 4.).

A MS will be introduced for each group in the following subsections(2.4.1 - 2.4.3). Those are:

\begin{itemize}
	\item \textbf{Conversion Service}\\
	This service will be responsible of the conversion from $img^{wsi}_i$ to $img^{dzi}_i$ (1.).
	\item \textbf{Annotation Service}\\
	This service will offer a GUI to view a $img^{dzi}_i$, as well as make and manage annotations (2. - 4.)
	\item \textbf{Tessellation Service}\\
	This service will be responsible for extracting a $ts_i$ from a given $A_i$ and $img^{dzi}_i$ (5.).
\end{itemize}


\subsection{Conversion Service}

The devices which create WSIs, so called \emph{whole slide scanners}, create images in various formats, depending on the producer and a lack of a defined standard\cite{Cornish13}. The Conversion Service (CS)\nmc{CS}{Conversion Service} has the goal of converting those formats to DZI\footnote{see chap. 1.2(i) and 1.2(ii) for the reasons}].

\begin{figure}[H]
	\begin{center}
		\includegraphics[scale=0.35]{img/processChainA.png}
		\caption{Visualization of the Conversion Service}
		\label{fig:fig2.1}
	\end{center}
\end{figure}

Upon invocation, the CS will take every single WSI inside a given directory and convert it to DZI. The output of each conversion will be saved in another specified folder. Valid image formats (and their corresponding producers) for conversion are:

\begin{itemize}
	\item .bif (Ventana)
	\item .mrxs (Mirax)
	\item .ndpi (Hamamatsu)
	\item .scn (Leica)
	\item .svs (Aperio)
	\item .svslide (Sakura)
	\item .tif (Aperio, Trestle, Ventana)
	\item .tiff (Philips)
	\item .vms (Hamamatsu)
	\item .vmu (Hamamatsu)
\end{itemize}


\subsection{Annotation Service}
As mentioned in 2.4, the Annotation Service (AS)\nmc{AS}{Annotation Service} will provide a graphical user interface (GUI)\nmc{GUI}{Graphical User Interface} to view a DZI, make annotations and manage those annotations. This also includes persisting made annotations in a file (see fig. 2.2).

\begin{figure}[H]
	\begin{center}
		\includegraphics[scale=0.25]{img/processChainB.png}
		\caption{Visualization of the Annotation Service}
		\label{fig:fig2.4}
	\end{center}
\end{figure}

The supplied GUI will offer different tools to help the user annotate the DZI, e.g. a ruler to measure the distance between 2 points. The annotations themselves will be made via drawing a contour around an object of interest and putting a specified label to that region. To ensure uniformity of annotations, labels will not be added in free text. Instead they will be selected from a predefined dictionary.


\subsection{Tessellation Service}

The task of the Tessellation Service (TS)\nmc{TS}{Tessellation Service} is to extract annotations and their corresponding image data in such a fashion that they will become usable as training samples for NN.

\begin{figure}[H]
	\begin{center}
		\includegraphics[scale=0.3]{img/processChainC.png}
		\caption{Visualization of the Tessellation Service}
		\label{fig:fig2.6}
	\end{center}
\end{figure}

Let there be a DZI $D$ and a corresponding set of annotations $A$. The TS will achieve the extraction by iterating over every $a_i \in A$, creating a sub-image $d_i$ which is the smallest bounding box around the region described by $a_i$ (see fig. 2.3). To be used as training sample, the TS must keep up the correspondence between $d_i$ and $a_i$.
\chapter{Conversion Service}

\section{Methodology}

The objective of the Conversion Service is to convert a given set of input WHIs into dzi files. Since a dzi is layered into a pyramid scheme, it is necessary to calculate the needed number of levels, as well as the dimensions of each level (see fig. 3.1 for an example).

\begin{figure}[H]
	\begin{center}
		\includegraphics[scale=0.5]{img/pyramid.png}
		\caption{Example of a pyramid scheme in image processing (source: http://iipimage.sourceforge.net/images/pyramid.png)}
		\label{fig:fig3.1}
	\end{center}
\end{figure}

Therefore, the Conversion Service must be able to open an WHI $img_{input}$ of any of the in 2.2.1 defined formats. Based on the size of $img_{input}$ the number of necessary levels $lvl$ must be calculated. Once $lvl$ has been determined, $img_{input}$ must be resized into an appropriate scale for each $lvl_i$ in $lvl$. The resized image will be called $img_i$, with $i$ representing the corresponding level. In the next step, every $img_i$ will be tessellated into $x*y$ tiles. Each tile will be referrenced via $t^i_{c,r}$, with $r$ being the row and $c$ being the column of the tile in $whi_i$. To complete the conversion, the Conversion Service must create a describing xml file for each converted image $img_{dzi}$.


\subsection{Creating a Deep Zoom Image}

To create a dzi, the Conversion Service must be capable of 

\begin{table}
	\begin{center}
		\begin{tabular}{| p{3cm} | p{4cm} | p{2cm} |}
			\hline
			\textbf{option} & \textbf{description} & \textbf{image format} \\
			\hline
      		Deep Zoom Composer & dekstop app for Windows & dzi \\ \hline
      		Image Composite Editor & panoramic image stitcher from Microsoft Research for the Windows desktop & dzi \\ \hline
      		DeepZoomTools.dll & .NET-library, comes with Deep Zoom Composer & dzi \\ \hline
      		deepzoom.py & Python & dzi \\ \hline
      		deepzoom & Perl utility & dzi \\ \hline
      		PHP Deep Zoom Tools & PHP & dzi \\ \hline
      		Deepzoom & PHP & dzi \\ \hline
     		 DZT & an image slicing library and tool written in Ruby & dzi \\ \hline
     		MapTiler &  desktop app for Windows, Mac, Linux & tms \\ \hline
      		VIPS & command line tool and library for a number of languages & dzi (via dzsave feature) \\ \hline
     		Sharp & Node.js, uses VIPS & dzi \\ \hline
      		MagickSlicer & shell script (Linux/Max) & dzi \\ \hline
      		Gmap Uploader Tiler & C++ & dzi \\ \hline
      		Node.js Deep Zoom Tools & Node.js, under construction & dzi \\ \hline
      		OpenSeaDragon DZI Online Composer & Web app (and PERL and PHP scripts) & dzi \\ \hline
      		Zoomable & service, offers embeds; no explicit API & dzi \\ \hline
      		ZoomHub & service, under construction & dzi \\ \hline
      		Kakadu & C++ library to encode or decode JPEG 2000 images & iiif \\ \hline
      		PyramidIO & Java (command line and library) & dzi \\ \hline
  		\end{tabular}
  	\caption{Overview of conversion options for zooming image formats (source: \cite{web:openseadragon})}
  	\end{center}
\end{table}
  
iiif\footnote{International Image Interoperability Framework (iiif), specified by the International Image Interoperability Framework group, is an image delivery API which responds to requests via HTTP and HTTPS \cite{web:iiif}}

tms\footnote{Tile Map Service (tms) is a tile scheme developed and maintained by the Open Source Geospatial Foundation \cite{web:tms}} 


\subsection{Deepzoom.py}

Deepzoom.py\footnote{see \url{https://github.com/openzoom/deepzoom.py} for further details} is a python script and part of Open Zoom\footnote{see \url{https://github.com/openzoom} for further details}. It can either be called directly over a terminal or imported as a module in another python script. The conversion procedure itself is analogous for both methods.

If run in a terminal the call looks like the following:

\begin{lstlisting}
	$ python deepzoom.py [options] [input file]
\end{lstlisting}

The various options and their default values can be seen in tab. 3.2. If called without a designated output destination, deepzoom.py will save the converted dzi right next to the original input file.

\begin{table}[H]
	\begin{center}
		\begin{tabular}{| r | l | r |}
			\hline
			\textbf{option} & \textbf{description} & \textbf{default} \\ \hline
			-h & show help dialog & - \\ \hline
			-d & output destination & - \\ \hline
			-s & size of the tiles in pixels & 254 \\ \hline
			-f & image format of the tiles & jpg\\ \hline
			-o & overlap of the tiles in pixels (0 - 10) & 1 \\ \hline
			-q & quality of the output image (0.0 - 1.0) & 0.8 \\ \hline
			-r & type of resize filter & antialias \\ \hline
		\end{tabular}
		\caption{Options for deepzoom.py}
	\end{center}
\end{table}

The resize filter is applied to interpolate the pixels of the image when changing its size for the different levels. Supported filters are:

\begin{itemize}
	\item cubic
	\item biliniear
	\item bicubic
	\item nearest
	\item antialias
\end{itemize}

When used as module in another python script, deepzoom.py can simply be imported via the usual \emph{import} command. To actually use deepzoom.py, a Deep Zoom Image Creator needs to be created. This class will manage the conversion process:

\begin{lstlisting}[frame=single]
# Create Deep Zoom Image Creator
creator = deepzoom.ImageCreator(tile_size=[size], 
	tile_overlap=[overlap],	tile_format=[format], 
	image_quality=[quality], resize_filter=[filter])
\end{lstlisting}

The options are analogous with the terminal version (compare tab. 3.2). To start the conversion process, the following call must be made within the python script:

\begin{lstlisting}[frame=single]
# Create Deep Zoom image pyramid from source
creator.create([source], [destination])
\end{lstlisting}

Upon calling, the ImageCreator opens the input image $img_{input}$ and creates a description with all the needed information for the dzis describing xml file\footnote{compare chap. 2.1.1}. After that, the number of levels is calculated. For this, the bigger value of height and width of $img_{input}$ is chosen (see eq. 3.1) and then used to determine the number of levels $lvl$ (see eq. 3.2).

\begin{equation}
	max{\textunderscore}dim = max(height, width)
\end{equation}

\begin{equation}
	lvl = {\lceil}log_2(max{\textunderscore}dim) + 1\rceil
\end{equation}

Once $lvl$ has been determined, $img_{input}$ will be resized in the chosen quality (-q/image{\textunderscore}quality) for every level $i$, with $i \in (0, lvl-1)$. The new resolution will be calculated for both dimensions $dim$ with a function $scale$ (see eq. 3.3) analogously. Furthermore, the image will be interpolated with the specified filter (-r/resize{\textunderscore}filter). The resized image will be called $img_i$.

\begin{equation}
	scale = {\lceil}dim * 0.5^{lvl-i}\rceil
\end{equation}

Once $img_i$ has been created, it will be tessellated into as many tiles of the specified size (-s/tile{\textunderscore}size) and with the specified overlap (-o/tile{\textunderscore}overlap) as possible. Since not every image will be of the size $2^n, n \in $ in either dimension, it is highly likely that the set of tiles for the last column/row will be smaller then specified in either dimension.

Every tile will be saved as [column]{\textunderscore}[row].[format] ([format] depending on -f/file{\textunderscore}format) in a folder called according to the corresponding level $i$. This folder will be located inside another folder, called [filename]{\textunderscore}files. The describing xml file will be persisted as [filename].dzi on the same level.


\section{Implementation}
\section{Test}
\subsection{Setup}
\subsection{Result}
\chapter{Annotation Service}

\section{Objective of the Annotation Service}
As described in 2.4.2, the goal of the AS is to provide a user with the possibility to:
\begin{enumerate}[(1)]
	\item view A WSI
	\item annotate a WSI
	\item manage made annotations
\end{enumerate}

In order to achieve objective (1) - (3), a GUI needs to be deployed which supports the user in working on those tasks. (3) also adds the need for file persistence management.


\section{Methodology}
As stated in 2.1, most vendors have proprietary image formats and their own implementation of a viewer for those, thus creating a vendor lock-in. Further do vendors often only support Windows platforms, ignoring other operating systems\cite{Cornish13}\cite{DICOM10}\cite{Farahanil15}. To avoid this, a solution must be found that is independent of operating system and vendor.

Chap. 3 already established a service to convert WSIs of various formats into the DZI format, solving the problem of multiple proprietary formats. 

Independence from an operating system can be achieved by using web technologies, especially when running an application in a web browser\cite{Tseytlin14}, since those are supported by all modern operating systems. 

Because of this, the AS will be implemented as a web browser application. 


\subsection{Functionality of the Annotation Service}
The goal of viewing a WSI (1) is a straight forward task. (2) and (3) are more elusive. For that reason, this subsection elaborates on the functionality needed to help achieve those objectives.

Annotations will be created by drawing directly onto the viewed WSI. If the user spots a region of interest, a contour can be drawn around it. This can  either be done in a free hand or polygon mode. In free hand mode, the contour will follow the mouse pointer until the mode is disabled again, all the while drawing a contour. Upon deactivation, the contour will be closed. In polygon mode, the user can place coordinates which will be connected from one to another in the order they're placed in. A contour in this mode considers to be closed, once a point on the contour is clicked a second time.


% draw free hand or draw polygon
% set poi
% measure distance
% create dictionaries
% activate differnt dictionaries
% add labels to dictionaries
% select labels
% why dictionaries and labels can only be deleted by manipulating the file on the server, but regions can be deleted?
% regions and context regions


\subsection{Parts of the Annotation Service}
The AS is implemented in 2 parts. Those are the Annotation Service Server (ASS)\nmc{ASS}{Annoation Service Server} and the Annotation Service Viewer (ASV)\nmc{ASV}{Annotation Service Viewer}.

This is because of the \emph{same-origin policy} (SOP)\nmc{SOP}{same-origin policy}. SOP is a security concept of the web application security model. It prevents a direct access to files, if the parent directory of the originating file is not an ancestor directory of the target file\cite{web:mdn}. Because of the SOP, WSIs would have to be located in the directory structure of the AS, which by itself doesn't create a problem. To get a new WSI there, however, the user would be forced to navigate through the structure of the AS, find the correct directory and then place it there manually. This makes knowledge of the service structure necessary and creates a horrible UX. Furthermore, tinkering with the file structure of the AS creates a possible source for errors.

A workaround of this problem is to deploy a web server, which can redirect the image request, access the WSI and return it in response\cite{Tseytlin14}. The use of DZI creates another advantage: the used image pyramid model reduces the network traffic necessary to load and show a WSI in a viewer\cite{Cornish13}\cite{DICOM10}.

Furthermore, even a single WSI takes up a lot of storage capacity\cite{Singh11}. Having multiple WSIs on a local hard drive would either create the need for huge amounts of available storage space or restrict the amount of accessible WSIs to a few at any given time. The latter solution would create two follow-up problems:
\begin{itemize}
	\item WSIs are medical images and as such confidential information. Therefore, not everyone is allowed to just have access to or copies of them\cite{COA}\cite{USSanDiego}. Once a copy of a WSI changes hands, it is virtually impossible to make sure that privacy regulations will be uphold.	
	\item With only a small amount out of all WSIs accessible at all times, the need for copying files back and forth arises as soon as the user wants to compare, update or correct a WSI, which is not on his local file system at the given moment. Not only is this a great source for possible errors, but also very time consuming and inefficient.
\end{itemize}

With the use of a web server as a central image repository, WSIs and the access to them can be managed in a centralized spot, while upholding confidentiality regulations. Furthermore, a user has access to all of her/his WSIs at any given time, without the need for creating subsets and copying files back and forth. Depending on the setup of the network, other factors can come into play as well. Access to and sharing of rare cases, educational material and training samples can be granted without a complicated distribution chain and a smaller risk for confidentiality issues. It also enables the consultation of case experts independent of their physical position on the planet\cite{Wilbur09}.


\subsubsection{Annotation Service Server}
The ASS has 2 main purposes.

First, it serves as a so called \emph{Digital Slide Repository} (DSR)\nmc{DSR}{Digital Slide Repository}. A DSR manages storage of WSIs and their metadata. Furthermore, it serves requested image data to a viewer client, such as the ASV\cite{Cornish13}. 

Second, it is responsible for file management. In detail, this means:
\begin{itemize}
	\item persist made annotations in a file
	\item deliver annotation data together with image data
	\item serve list of all available label dictionaries
	\item serve label dictionary entries
	\item save added entries to existing label dictionary
	\item create new label dictionaries
\end{itemize}




% local for now


\subsubsection{Annotation Service Viewer}
To enable the pathologist to view a WSI and annotate it, a GUI needs to be deployed. This GUI will be called Annotation Service Viewer and developed in an iterative approach with the help of selected pathologists. After each iteration, the GUI and user experience (UX)\nmc{UX}{User Experience} will be evaluated. This way, the ASV can be adapted to the needs of a real life environment based on the pathologists feedback.

\begin{figure}[H]
	\begin{center}
		\includegraphics[scale=0.2]{img/microdrawUI.png}
		\caption{Microdraw GUI with opened WSI}
		\label{fig4_microdrawUI}
	\end{center}
\end{figure}

The first iteration of the ASV will be based on an open source project called \emph{MicroDraw}\footnote{See \url{https://github.com/r03ert0/microdraw} for more information on the MicroDraw project} (see fig. \ref{fig4_microdrawUI} for MicroDraws GUI).  MicroDraw is a web application to view and annotate \emph{"high resolution histology data"}\cite{web:microdraw2}. The visualization is based on another open source project, called \emph{OpenSeadragon}\cite{web:openseadragon}. Annotations are made possible by the use of \emph{Paper.js}\footnote{see \url{http://paperjs.org/} for more information on Paper.js}. Apart from the frameworks used, MicroDraw is written in JavaScript using HTML5, CSS3 and jQuery\footnote{see \url{https://jquery.com/} for more information on jQuery}.


\section{Implementation}
% what was used for implementation
\subsection{Technologies and Frameworks}
% openslide, openslide web server
% python web server, flask, openslide, java script, html5, css, jquery
\subsection{Annotation Service Viewer}
% documentation
\subsection{Annotation Service Server}
% documentation
\chapter{Tessellation Service}

\section{Objective of the Tessellation Service}
\label{sec5_objective}
The objective of the TS is to provide a service that extracts image regions annotated by the AS to create samples for the training of a NN\footnote{
	Compare subsection \ref{sec2_learning}
}. In this context, "extraction" describes the process of creating sub images of the original image, which contain all of the annotated ROI and as little of anything else as possible. Additionally, the correspondence with the associated region label must be kept. The AS uses \emph{regions}\footnote{
	Compare subsection \ref{sec4_region}
} to describe ROIs. Those regions are persisted in a JSON file.

Therefore, the TS has 2 objectives:
\begin{enumerate}[(1)]
	\item parse a JSON file and acquire its region data
	\item extract ROIs based on the acquired region data
\end{enumerate}


\section{Methodology}
\label{sec5_method}
The objective of the TS is to create usable training samples for NNs, as stated in section \ref{sec5_objective}. Depending on the setup, chosen learning method\footnote{
	Compare section \ref{sec2_introNN}
} and purpose of the NN, the requirements imposed on the training samples may vary. Smith demonstrates in \cite{Smith97} exemplary how to train a NN to recognize letters in images of written text by training it with 10x10 pixel grayscale images (256 gray levels/pixel) of letters. Shereena and David introduce a novel content based image retrieval classification method in \cite{Shereena14}, based on color or texture features. Other approaches extract features through the use of mathematical models from the supplied images (such as edges or shapes)\cite{Harvey91}.

Because of those varying requirements, the TS will be capable of producing different output:
\begin{enumerate}[(1)]
	\item Unaltered image of ROI
	\item Resize images to a specific width and height
	\item Approximate ROIs via tessellation
	\item Convert extracted images to grayscale
\end{enumerate}

A user can draw a region's path without any restrictions concerning the pattern, resulting in ROIs that can be of arbitrary shape. Therefore, bounding boxes (BB)\nmc{BB}{Bounding Box}\footnote{
	A BB is a rectangular body, fully enclosing a provided (two dimensional) object of arbitrary shape\cite{Toussaint83}.
} are used for ROI extraction in the cases (1) and (2) (see fig. \ref{fig5_bbExample} for an example).

\begin{figure}[H]
	\begin{center}
		\includegraphics[scale=0.6]{img/bb1.png}
		\caption{Examplary BBs: $B_1$ is BB of $M_1$, $B_2$ is BB of $M_2$ (source: \url{http://www.idav.ucdavis.edu/education/GraphicsNotes/Bounding-Box/Bounding-Box.html})}
		\label{fig5_bbExample}
	\end{center}
\end{figure}

In the case of (1), an ROI's BB is copied pixel by pixel into a new image. The resizing (scaling the output image up or down to the provided pixel values) in the case of (2) makes preprocessing of the BB necessary. This has 2 reasons:
\begin{itemize}
	\item If the aspect ration of the provided width and height is different than the one of the BB, the resulting image will be distorted.
	\item When scaling images down, interpolation can be used, to reduce the information loss of the image\cite{Thevanez00}. This is partially possible when scaling images up as well, e.g. via fractal interpolation, but a non trivial task\cite{Guerdri16}.
\end{itemize}

Therefore, the size of the BB will be adjusted to match the aspect ratio of the provided width and height. If the resulting BB is bigger then the provided width and height, the image will be scaled down and interpolated. If the BB is smaller, it will be scaled up instead of the image. This leads to a bigger area inside the BB that is not part of the ROI, but a pixel ratio of 1:1 between original and extracted image, resulting in no distortion or loss of quality.

Case (3) approximates a ROI by tessellating\footnote{
	\emph{Tessellation} describes the tiling of a plane using one or more geometric shapes with no overlaps or gaps\cite{Clifford09}.
} it into tiles of the provided width and height (see fig. \ref{fig5_tesExample} for an example). Every tile inside the region's enclosing path (or touched by it) is targeted by the extraction. Each tile is extracted into an individual image.

\begin{figure}[H]
	\begin{center}
		\includegraphics[scale=0.5]{img/tessellation.png}
		\caption{Example of an ROI approximated via tessellation (gray tiles belong to the ROI)}
		\label{fig5_tesExample}
	\end{center}
\end{figure}

Cases (1) - (3) can be combined with (4) to convert the extracted images into grayscale images.

\section{Implementation}
\label{sec5_impl}
The TS is implemented as a python script called \textbf{\emph{TessellationService.py}}\footnote{
	See its GitHub repository for more information: \url{https://github.com/SasNaw/TessellationService}.
}. \emph{OpenSlide} was used to access and read a provided WSI, as in chapters \ref{sec3_cs} (VIPS uses OpenSlide for the same reason\cite{cupitt96}) and \ref{sec4_as}. Additionally, \emph{NumPy}\footnote{
	See \url{http://www.numpy.org/} for more information on NumPy.
} and \emph{OpenCV}(Open Source Computer Vision Library)\nmc{OpenCV}{Open Source Computer Vision Library}\footnote{
	See \url{http://opencv.org/} for more information on OpenCV.
} were used to increase imaging and calculating functionality.

NumPy is a python library for efficient scientific computing, especially concerning multi-dimensional array objects\cite{Walt11}.

OpenCV  is an open source computer vision and machine learning software library. It was built to provide a common infrastructure for computer vision applications. It offers numerous functions regarding image processing and machine learning, with a strong focus on real-time computing and efficiency\cite{Bradski08}.

The TS can be called over the python interpreter:

\begin{lstlisting}
	$ python TessellationService [input] [params]
\end{lstlisting}

The files which are supposed to be targeted by the extraction process are handed to the TS via the \emph{[input]} argument. This can be
\begin{itemize}
	\item a WSI of proprietary format
	\item a DZI
	\item a directory with WSIs and DZIs
	\item a list of all the types mentioned above
\end{itemize}

If a directory is handed to the TS, all subdirectories will be searched as well.

Each WSI and DZI targeted by the TS must have an associated JSON file with annotations created by the AS\footnote{
	Compare \ref{sec4_as}
}. The [input] argument is mandatory.

As mentioned in section \ref{sec5_method}, it might be necessary to create different output images for different purposes. Therefore, the TS has a list of optional parameters to manipulate the created output in order to be applicable to a wider variety of cases (see tab. \ref{tab5_tsParams}). Those parameters are optional. 

\begin{table}[H]
	\begin{center}
		\begin{tabular}{| p{3cm} | p{5cm} | p{3cm} |}
			\hline
			\textbf{name} & \textbf{description} & \textbf{default}\\ \hline
			-h, --help & show help & - \\ \hline
			-f, --force-overwrite & flag to overwrite images with the same name (if not supplied, \emph{"{\textunderscore}copy"} will be added to the image name) & false \\ \hline
			-g, --grayscale & flag to convert images to grayscale images before saving them & false \\ \hline
			-i, --interpolation & choose interpolation method: nearest neighbor, bilinear, bicubic, lanczos  & nearest neighbor interpolation \\ \hline
			-o, --output [directory] & choose output directory (if not provided, the TS will save all extracted images in its own directory) & - \\ \hline
			-r, --resize [width] [height] & resize all output images to the provided [width] and [height] in pixel & - \\ \hline
			-s, --show-tessellated & flag to create window and show stitched image resulting from the tessellation process (only for debugging purposes, see \emph{"-t"}) & false\\ \hline
			-t, --tessellate [width] [height] & tessellate regions into tiles of the provided [width] and [height] in pixels & - \\ \hline
		\end{tabular}
		\caption{Optional parameters for the TS}
		\label{tab5_tsParams}
	\end{center}
\end{table}


\subsection{Extraction process}


\begin{figure}[H]
	
	\begin{center}
		\includegraphics[scale=0.45]{img/ts_run.png}
		\caption{Activity diagram of TS' extraction process}
		\label{fig5_extractionProcess}
	\end{center}
\end{figure}


\subsubsection{WSI}

\begin{figure}[H]
	\begin{center}
		\includegraphics[scale=0.5]{img/ts_wsi_uml.png}
		\caption{Activity diagram of TS' WSI extraction}
		\label{fig5_tsWsiUml}
	\end{center}
\end{figure}


\subsubsection{DZI}

\begin{figure}[H]
	\begin{center}
		\includegraphics[scale=0.4]{img/ts_dzi_uml.png}
		\caption{Activity diagram of TS' DZI extraction}
		\label{fig5_tsDziUml}
	\end{center}
\end{figure}


\subsubsection{Tessellation}


\section{Test}
\subsection{Setup}
\subsection{Result}
\chapter{Conclusion}
\section{Results}
\section{Conclusion}
\section{Future tasks}
% turn ASS into a real server
% deliver a working segmentation script
% implement wsi browser
% implement wsi stack in viewer?
% integrate CS and TS into ASV
% visualization of tessellation in ASV
\begin{appendices}
	\chapter{Test Data}

\section{Files on disc}
\label{secA_cd}

A disc is included at the end of this thesis. This disc contains the 3 introduced services from the chapters \ref{sec3_cs} (CS), \ref{sec4_as} (AS) and \ref{sec5} (TS). Alternatively, they are available at their corresponding GitHub repositories (see tab. \ref{tabA_paths}).

\begin{table}[H]
	\begin{center}
		\begin{tabular}{| p{1.5cm} | p{4.5cm} | p{5cm} |}
			\hline
			\textbf{service} & \textbf{disc path} & \textbf{GitHub repository} \\ \hline
			CS & /services/ConversionService & \url{https://github.com/SasNaw/ConversionService} \\ \hline
			AS & /services/AnnotationService & \url{https://github.com/SasNaw/AnnotationService} \\ \hline
			TS & /services/TessellationService & \url{https://github.com/SasNaw/TessellationService}  \\ \hline
		\end{tabular}
		\caption{Services with their disc paths and repository URLs}
		\label{tabA_paths}
	\end{center}
\end{table}

All tools presented in this thesis are open source and their use, further development and diversification are highly encouraged.

The disc also contains 6 exemplary WSI files (DZI, MRXS, NDPI, SCN, SVS, TIF). Due to the limited storage space available on the medium and the size of most WSI files, not all test files could be provided on disc. They can be found under \emph{/testdata/wsi/} and are usable for all 3 services.

The JSON save file from \ref{sec5_test} can also be found on the disc, at \emph{/testdata/{\allowbreak}json/CMU-1.svs{\textunderscore}example.json}.



\section{Free Whole Slide Images}
\label{secA}
The following test data was used for all tests involving CS, AS and TS. It can be found at OpenSlide's homepage, at the freely distributable test data section\footnote{
	See \url{http://openslide.cs.cmu.edu/download/openslide-testdata/} for more information.
}. Various slides can be found there. The sections \ref{sec_A1} - \ref{sec_A8} give listings of all used WSIs, sorted by vendor and file format.

\subsection{Aperio (.svs)}
\label{sec_A1}

\begin{table}[H]
	\begin{center}
		\begin{tabular}{| p{4cm} | p{2cm} | p{5cm} |}
			\hline
			\textbf{name} & \textbf{size (MB)} & \textbf{description} \\ \hline
			CMU-1-JP2K-33005.svs & 126.42 & Export of CMU-1.svs, brightfield, JPEG 2000, RGB\\ \hline
			CMU-1-Small-Region.svs & 1.85 & Exported region from CMU-1.svs, brightfield, JPEG, small enough to have a single pyramid level \\ \hline
			CMU-1.svs & 169.33 & Brightfield, JPEG \\ \hline
			CMU-2.svs & 372.65 & Brightfield, JPEG \\ \hline	
			CMU-3.svs & 242.06 & Brightfield, JPEG \\ \hline	
			JP2K-33003-1.svs & 60.89 & Aorta tissue, brightfield, JPEG 2000, YCbCr \\ \hline
			JP2K-33003-2.svs & 275.85 & Heart tissue, brightfield, JPEG 2000, YCbCr  \\ \hline
		\end{tabular}
		\caption{Aperio data set (source: \url{http://openslide.cs.cmu.edu/download/openslide-testdata/Aperio/})}
	\end{center}
\end{table}


\subsection{Generic Tiled tiff (.tiff)}

\begin{table}[H]
	\begin{center}
		\begin{tabular}{| p{4cm} | p{2cm} | p{5cm} |}
			\hline
			\textbf{name} & \textbf{size (MB)} & \textbf{description} \\ \hline
			CMU-1.tiff & 194.66 & Conversion of CMU-1.svs to pyramidal tiled TIFF, brightfield \\ \hline
		\end{tabular}
		\caption{Generic Tiled tiff data set (source: \url{http://openslide.cs.cmu.edu/download/openslide-testdata/Generic-TIFF/})}
	\end{center}
\end{table}


\subsection{Hamamatsu (.ndpi)}
\label{secA_Hama}

\begin{table}[H]
	\begin{center}
		\begin{tabular}{| p{4cm} | p{2cm} | p{5cm} |}
			\hline
			\textbf{name} & \textbf{size (MB)} & \textbf{description} \\ \hline
			CMU-1.ndpi & 188.86 & Small scan with valid JPEG headers, brightfield, circa 2009 \\ \hline
			CMU-2.ndpi & 382.14 & Brightfield, circa 2009 \\ \hline
			CMU-3.ndpi & 270.1 & Brightfield, circa 2009 \\ \hline
			OS-1.ndpi & 1,860 & H\&E stain, brightfield, circa 2012 \\ \hline
			OS-2.ndpi & 931.42 & Ki-67 stain, brightfield, circa 2012 \\ \hline
			OS-3.ndpi & 1,370 & PTEN stain, brightfield, circa 2012 \\ \hline
		\end{tabular}
		\caption{Hamamatsu data set (.ndpi, source: \url{http://openslide.cs.cmu.edu/download/openslide-testdata/Hamamatsu/})}
	\end{center}
\end{table}


\subsection{Hamamatsu (.vms)}

\begin{table}[H]
	\begin{center}
		\begin{tabular}{| p{4cm} | p{2cm} | p{5cm} |}
			\hline
			\textbf{name} & \textbf{size (GB)} & \textbf{description} \\ \hline
			CMU-1.zip & 0.62 & Brightfield \\ \hline
			CMU-2.zip & 1.13 & Brightfield \\ \hline
			CMU-3.zip & 0.91 & Brightfield \\ \hline
		\end{tabular}
		\caption{Hamamatsu data set (.vms, source: \url{http://openslide.cs.cmu.edu/download/openslide-testdata/Hamamatsu-vms/})}
	\end{center}
\end{table}


\subsection{Leica (.scn)}

\begin{table}[H]
	\begin{center}
		\begin{tabular}{| p{4cm} | p{2cm} | p{5cm} |}
			\hline
			\textbf{name} & \textbf{size (GB)} & \textbf{description} \\ \hline
			Leica-1.scn & 0.28 & Brightfield, single ROI, 2010/10/01 schema \\ \hline
			Leica-2.scn & 2.1 & Mouse kidney, H\&E stain, brightfield, multiple ROIs with identical resolutions, 2010/10/01 schema \\ \hline
			Leica-3.scn	 & 2.79 & Mouse kidney, H\&E stain, brightfield, multiple ROIs with different resolutions, 2010/10/01 schema \\ \hline
			Leica-Fluorescence-1.scn & 0.02 & Fluorescence, 3 channels, single ROI, 2010/10/01 schema \\ \hline
		\end{tabular}
		\caption{Leica data set (source: \url{http://openslide.cs.cmu.edu/download/openslide-testdata/Leica/})}
	\end{center}
\end{table}


\subsection{Trestle (.tiff)}

\begin{table}[H]
	\begin{center}
		\begin{tabular}{| p{4cm} | p{2cm} | p{5cm} |}
			\hline
			\textbf{name} & \textbf{size (MB)} & \textbf{description} \\ \hline
			 CMU-1.zip & 158.87 & Brightfield \\ \hline
			 CMU-2.zip & 304.22 & Brightfield \\ \hline
			 CMU-3.zip & 223.11 & Brightfield \\ \hline
		\end{tabular}
		\caption{Trestle data set (source: \url{http://openslide.cs.cmu.edu/download/openslide-testdata/Trestle/})}
	\end{center}
\end{table}


\subsection{Ventana (.bif)}
\begin{table}[H]
	\begin{center}
		\begin{tabular}{| p{4cm} | p{2cm} | p{5cm} |}
			\hline
			\textbf{name} & \textbf{size (GB)} & \textbf{description} \\ \hline
			 OS-1.bif & 3.61 & H\&E stain, brightfield \\ \hline
			 OS-2.bif & 2.53 & Ki-67 stain, brightfield \\ \hline
 		\end{tabular}
		\caption{Trestle data set (source: \url{http://openslide.cs.cmu.edu/download/openslide-testdata/Trestle/})}
	\end{center}
\end{table}

\subsection{Mirax (.mrxs)}
\label{sec_A8}
\begin{table}[H]
	\begin{center}
		\begin{tabular}{| p{4cm} | p{2cm} | p{5cm} |}
			\hline
			\textbf{name} & \textbf{size (GB)} & \textbf{description} \\ \hline
			CMU-1-Exported.zip & 2.02 & Export of CMU-1.mrxs with overlaps resolved, brightfield, JPEG, CURRENT{\textunderscore}SLIDE{\textunderscore}VERSION 2.3 \\ \hline
			CMU-1-Saved-1{\textunderscore}16.zip & 0.003 & Quick save of CMU-1.mrxs at 1/16 resolution (multiple positions per image), brightfield, JPEG, CURRENT{\textunderscore}SLIDE{\textunderscore}VERSION 1.9 \\ \hline
			CMU-1-Saved-1{\textunderscore}2.zip & 0.14 & Quick save of CMU-1.mrxs at 1/2 resolution (multiple images per position), brightfield, JPEG, CURRENT{\textunderscore}SLIDE{\textunderscore}VERSION 1.9 \\ \hline
			CMU-1.zip & 0.54 & Brightfield, JPEG, CURRENT{\textunderscore}SLIDE{\textunderscore}VERSION 1.9 \\ \hline
			CMU-2.zip & 1.22 & Brightfield, JPEG, CURRENT{\textunderscore}SLIDE{\textunderscore}VERSION 1.9 \\ \hline
			CMU-3.zip & 0.65 & Brightfield, JPEG, CURRENT{\textunderscore}SLIDE{\textunderscore}VERSION 1.9 \\ \hline
			Mirax2-Fluorescence-1.zip & 0.06 & Fluorescence, 3 channels, JPEG, CURRENT{\textunderscore}SLIDE{\textunderscore}VERSION 2 \\ \hline
			Mirax2-Fluorescence-2.zip & 0.04 & Fluorescence, 3 channels, JPEG, CURRENT{\textunderscore}SLIDE{\textunderscore}VERSION 2 \\ \hline
			Mirax2.2-1.zip & 2.61 & HPS stain, brightfield, JPEG, CURRENT{\textunderscore}SLIDE{\textunderscore}VERSION 2.2 \\ \hline
			Mirax2.2-2.zip & 2.38 & HPS stain, brightfield, JPEG, CURRENT{\textunderscore}SLIDE{\textunderscore}VERSION 2.2	\\ \hline
			Mirax2.2-3.zip & 2.77 & HPS stain, brightfield, JPEG, CURRENT{\textunderscore}SLIDE{\textunderscore}VERSION 2.2	\\ \hline
			Mirax2.2-4-BMP.zip & 0.95 & Brightfield, BMP, CURRENT{\textunderscore}SLIDE{\textunderscore}VERSION 2.2	\\ \hline
			Mirax2.2-4-PNG.zip & 1.01 & Brightfield, PNG, CURRENT{\textunderscore}SLIDE{\textunderscore}VERSION 2.2 \\ \hline
		\end{tabular}
		\caption{Mirax data set (source: \url{http://openslide.cs.cmu.edu/download/openslide-testdata/Mirax/})}
	\end{center}
\end{table}
	\chapter{Annotation Service Documentation}

The following two sections document the implemented functions of the ASS (section \ref{sec_B1}) and ASV (section \ref{sec_B2}) in detail. Both files can be found in the AS' repository at:

 \url{https://github.com/SasNaw/AnnotationService}.

\section{Annotation Service Server}
\label{sec_B1}

\subsubsection{index{\textunderscore}dzi()}
If the client requests a DZI (URL ends in \emph{".dzi"}), \texttt{index{\textunderscore}dzi()} renders an ASV and passes the necessary information (slide URL, file name, MPP) to it.

It builds the file name and slide URL (line 3 and 4) for a requested DZI. A metadata.txt will be present in the [slide name]{\textunderscore}files directory, if the DZI was created with the CS. If so, the function will try to fetch the metadata information about MPP and calculate the average height of a pixel (line 6 - 16). If the MPP metadata could not be fetched, it is set to 0 (line 17 - 18). File name, URL and MPP are then passed onto the ASV, which then is rendered with the given information (line 19).

\begin{lstlisting}[language=Python, frame=single]
@app.route('/wsi/<path:file_path>.dzi')
def index_dzi(file_path):
	file_name = file_path + '.dzi'
	slide_url = '/wsi/' + file_name
	# read dzi file
	try:
		with open('static/wsi/' + file_path + '_files/metadata.txt') as file:
			mpp_x = 0
			mpp_y = 0
			metadata = file.read().split('\n')
			for property in metadata:
				if openslide.PROPERTY_NAME_MPP_X in property:
					mpp_x = property.split(': ')[1]
				elif openslide.PROPERTY_NAME_MPP_Y in property:
					mpp_y = property.split(': ')[1]
			slide_mpp = (float(mpp_x) + float(mpp_y)) / 2
	except IOError:
		slide_mpp = 0
	return render_template('as_viewer.html', slide_url=slide_url, slide_mpp=slide_mpp, file_name=file_name)
\end{lstlisting}


\subsubsection{index{\textunderscore}wsi()}
When the client requests a proprietary WSI (URL \emph{does not} end in ".dzi"), \texttt{index{\textunderscore}wsi()} renders an ASV and passes the necessary information (slide URL, file name, MPP) to it. Furthermore, it wraps a DZG around the proprietary WSI and adds that to the WSGI object.

Line 22 - 27 create a map with the optional DZG parameters (compare tab. \ref{tab4_DZGparam}) and turn them into a dictionary. Line 28 reads the proprietary WSI. A DZG with the supplied parameters\footnote{Compare tab. \ref{tab4_assParams}} is created, which wraps the proprietary slide object to add Deep Zoom support (line 29 - 31). The created DZG is added to the WSGI object (line 29). Line 32 - 37 fetch associated images, the metadata (line 33), wrap the associated images with a DZG of their own and add this, together with the metadata, to the WSGI object. Line 39 - 43 fetch the MPP metadata and calculate the average MPP (or set it to 0, if not found). 
Line 44 creates a URL for the DZG object with Flasks \texttt{url{\textunderscore}for(endpoint, **values)} function. This URL is passed, together with the MPP and file path, to an ASV which then gets rendered (line 45).

\begin{lstlisting}[language=Python, frame=single]
@app.route('/wsi/<path:file_path>')
def index_wsi(file_path):
	config_map = {
		'DEEPZOOM_TILE_SIZE': 'tile_size',
		'DEEPZOOM_OVERLAP': 'overlap',
		'DEEPZOOM_LIMIT_BOUNDS': 'limit_bounds',
	}
	opts = dict((v, app.config[k]) for k, v in config_map.items())
	slide = open_slide('static/wsi/' + file_path)
	app.slides = {
		SLIDE_NAME: DeepZoomGenerator(slide, **opts)
	}
	app.associated_images = []
	app.slide_properties = slide.properties
	for name, image in slide.associated_images.items():
		app.associated_images.append(name)
		slug = slugify(name)
		app.slides[slug] = DeepZoomGenerator(ImageSlide(image), **opts)
	try:
		mpp_x = slide.properties[openslide.PROPERTY_NAME_MPP_X]
		mpp_y = slide.properties[openslide.PROPERTY_NAME_MPP_Y]
		slide_mpp = (float(mpp_x) + float(mpp_y)) / 2
	except (KeyError, ValueError):
		slide_mpp = 0
	slide_url = url_for('dzi', slug=SLIDE_NAME)
	return render_template('as_viewer.html', slide_url=slide_url, slide_mpp=slide_mpp, file_name=file_path)
\end{lstlisting}


\subsubsection{dzi(slug)}
If \texttt{index{\textunderscore}wsi()} was called before, a URL was generated for the WSI. This URL will be requested from the ASS by OpenSeadragon, which causes \texttt{slug(dzi)} to be called. \texttt{slug(dzi)} creates the DZI metadata and returns it to OpenSeadragon.

The \emph{dzi} parameter is the slide URL generated in \texttt{index{\textunderscore}wsi} (line 44).

Line 48 retrieves the format for the individual Deep Zoom tiles. Line 49 - 52 try to create a response. If a response can not be created, because the requested DZG is unknown, a "404 Not Found" http status code will be returned instead. If the DZG could be found, a response with the DZIs metadata will be created via the DZGs \texttt{get{\textunderscore}dzi(format)} function (line 50, compare subsection \ref{sec4_openslide}).

\begin{lstlisting}[language=Python, frame=single]
@app.route('/<slug>.dzi')
def dzi(slug):
	format = app.config['DEEPZOOM_FORMAT']
	try:
		resp = make_response(app.slides[slug].get_dzi(format))
		resp.mimetype = 'application/xml'
		return resp
	except KeyError:
		# Unknown slug
		abort(404)
\end{lstlisting}


\subsubsection{tile(slug, level, col, row, format)}
If a response for OpenSeadragon was created via \texttt{slug(dzi)}, OpenSeadragon will request the individual image tiles in such a way, that, through the use of the route() decorator, \texttt{tile(slug, level, col, row, format)} will be called.

As in \texttt{slug(dzi)}, the \emph{slug} parameter is the slide URL generated in \texttt{index{\textunderscore}wsi} (line 44). The parameters \emph{level}, \emph{col} and \emph{row} describe the DZI level and address of the requested image tile. \emph{format} is the image format of the tile.

If the format is not JPEG or PNG, the ASS return a "404 Not Found" http status code (line 58 - 61).

If the format is either JPEG or PNG, the requested tile is generated through the use of the DZGs \texttt{get{\textunderscore}tile(level, address)} function (line 63). If it was not possible to generate the tile, a "404 Not Found" http status code will be returned.

The generated tile is then saved into a PIL image object\footnote{See \url{http://pillow.readthedocs.io/en/3.3.x/reference/Image.html}}, stored in either a JPEG or PNG image and returned as response to OpenSeadragon (line 70 - 74).

\begin{lstlisting}[language=Python, frame=single]
@app.route('/<slug>_files/<int:level>/<int:col>_<int:row>.<format>')
def tile(slug, level, col, row, format):
	format = format.lower()
	if format != 'jpeg' and format != 'png':
		# Not supported by Deep Zoom
		abort(404)
	try:
		tile = app.slides[slug].get_tile(level, (col, row))
	except KeyError:
		# Unknown slug
		abort(404)
	except ValueError:
		# Invalid level or coordinates
		abort(404)
	buf = PILBytesIO()
	tile.save(buf, format, quality=app.config['DEEPZOOM_TILE_QUALITY'])
	resp = make_response(buf.getvalue())
	resp.mimetype = 'image/%s' % format
	return resp
\end{lstlisting}


\subsubsection{saveJson()}
When the client sends JSON data to save, the \texttt{saveJson()} function is called.

The associated request is a POST request. This means that the posted data needs to be extracted. This can be done via Flasks \emph{request object}\footnote{Compare subsection \ref{sec4_flask}} (line 77 - 79). The file path will be transmitted as \emph{"source"}, the content to save as \emph{"json"}.

If there is something to save (line 80), the content will be written into the provided file. If the file does not exist yet, it will be created (line 81 - 82).

\begin{lstlisting}[language=Python, frame=single]
@app.route('/saveJson', methods=['POST'])
def saveJson():
	dict = request.form
	source = dict.get('source', default='')
	json = dict.get('json', default='{}').encode('utf-8')
	if len(source) > 0:
		with open('static/' + source, 'w+') as file:
			file.write(json)
	return 'Ok'
\end{lstlisting}


\subsubsection{loadJson()}
When the client requests JSON data, \texttt{loadJson()} is called.

The source of the JSON data is passed in the URL as parameter (\emph{"src=[path to source]"}). The src parameter can be extracted via Flasks request object\footnote{Compare subsection \ref{sec4_flask}} (line 86). If the provided source is a file, the content will be read and returned as JSON data (line 87 - 90). Otherwise an empty JSON list is returned (line 91 - 92).

\begin{lstlisting}[language=Python, frame=single]
@app.route('/loadJson')
def loadJson():
	source = 'static/wsi/' + request.args.get('src', '')
	if os.path.isfile(source):
		with open(source, 'r') as file:
			content = file.read()
			return jsonify(content)
	else:
		return jsonify('[]')
\end{lstlisting}


\subsubsection{createDictionary()}
When the client requests the creation of a new dictionary, \texttt{createDictionary()} is called. The name of the new dictionary is passed as URL parameter (\emph{name= [name]}). The name parameter can be extracted via Flasks request object\footnote{Compare subsection \ref{sec4_flask}} (line 95).

Once the name was extracted, the function checks if a dictionary with the provided name already exists. If so, \emph{"error"} is returned (line 96 - 99). Otherwise a new, empty dictionary is created (line 101 - 102). To switch to the newly created dictionary, the configuration file must be updated (line 103 - 107).

As response, the name and path of the newly created dictionary is returned (line 108 - 109).

\begin{lstlisting}[language=Python, frame=single]
@app.route('/createDictionary')
def createDictionary():
	name = request.args.get('name', '')
	path = 'static/dictionaries/' + name
	if os.path.isfile(path):
		# dictionary already exists
		return 'error'
	else:
		with open(path, 'w+') as dictionary:
			dictionary.write("[]")
		with open('static/configuration.json', 'r') as config:
			content = json.loads(config.read())
			content['dictionary'] = name
		with open('static/configuration.json', 'w+') as config:
			config.write(json.dumps(content))
		respone = '{"name":"' + name + '", "path":"/' + path + '"}'
		return respone
\end{lstlisting}


\subsubsection{getDictionaries()}
The \texttt{getDictionaries()} function is called, when the client requests a list of all available dictionaries.

If no dictionaries could be found, "-1" will be returned, otherwise a JSON list of all available dictionaries.

\begin{lstlisting}[language=Python, frame=single]
def getDictionaries():
	dir = 'static/dictionaries/'
	if os.path.isfile(dir):
		# no dictionaries found
		return '-1'
	else:
		# return dictionaries
		return json.dumps(os.listdir(dir))
\end{lstlisting}


\section{Annotation Service Viewer}
\label{sec_B2}



\begin{lstlisting}[language=JavaScript, frame=single]

\end{lstlisting}
	\chapter{Tessellation Service Documentation}
\label{secC}
The TS is openly available at GitHub and comes with instructions for usage and installation. The Repository can be found at:

\url{https://github.com/SasNaw/TessellationService}

\section{Main}

\subsubsection{run(input)}
As stated in section \ref{sec5_method}, each input element can be a file or a dictionary. Therefore, the individual entries must be examined. The \texttt{run(input)} function does exactly that. Each element's type (file or directory) is checked. If it is a file, the extraction process is started (line 8). If it is a directory, its files and subdirectories are extracted (line 5).

\begin{lstlisting}[frame=single,language=python]
def run(input):
	for element in input:
		# input is folder:
		if(os.path.isdir(element)):
			files_from_dir(element)
		# input is file:
		elif(os.path.isfile(element)):
			regions_from_file(element)
\end{lstlisting}


\subsubsection{files{\textunderscore}from{\textunderscore}dir(element)}
\texttt{files{\textunderscore}from{\textunderscore}dir(element)} gets the content of the provided directory (line 4). For every subdirectory found, \texttt{files{\textunderscore}from{\textunderscore}dir(element)} is called recursively, until the end of each directory tree is reached. An exception from this are the DZI \emph{"{\textunderscore}files"} directories, since they only contain the corresponding DZI's tessellated image levels (line 6 - 8). If a file was found, the extraction process is started for it (line 9 - 10).

\begin{lstlisting}[frame=single,language=python]
def files_from_dir(dir):
	if not dir.endswith('/'):
		dir = dir + '/'
	contents = os.listdir(dir)
	for content in contents:
		if os.path.isdir(dir + content):
			if not content.endswith('_files'):
				files_from_dir(dir + content)
		else:
			regions_from_file(dir + content)
\end{lstlisting}


\subsubsection{regions{\textunderscore}from{\textunderscore}file(element)}
Since the implementation for WSI and DZI differs, a distinction must be made, which of two approaches is chosen. \texttt{regions{\textunderscore}from{\textunderscore}file(element)} makes this distinction and starts the extraction process for the provided element. If the provided element is not of the DZI type, it is checked if it is a valid WSI (via \texttt{is{\textunderscore}supported(file)}, see line 5) before the process is started. This way, only DZI and WSI will be targeted with extraction attempts.
\begin{lstlisting}[frame=single,language=python]
def regions_from_file(file):
	if file.endswith('.dzi'):
		dzi(file)
	else:
		if(is_suppoted(file)):
			wsi(file)
\end{lstlisting}


\section{WSI}

\subsubsection{wsi(file)}
\texttt{wsi(file)} extracts the regions from a WSI. To do so, the provided file must have an associated JSON file which was annotated with the dictionary provided as argument when the TS was started\footnote{
	Compare subsection \ref{sec5_exec}
}.

OpenSlide is used to read the provided WSI (see line 2). The file name is extracted from the file path (see line 3). The saved regions are loaded and parsed via \texttt{read{\textunderscore}json(path)} function (see line 4).

For each parsed region, an image is extracted together with a metadata file. If the tessellation parameter was provided when TessellationService.py was called, \texttt{tessellate{\textunderscore}wsi(slide, slide{\textunderscore}name, region)} is called. Otherwise, the BB is calculated via \texttt{get{\textunderscore}bounding{\textunderscore}box(region)}. If the -r parameter was provided, the BB will be adjusted to fit the supplied image ratio (via \texttt{resize{\textunderscore}bounding{\textunderscore}box(bounding{\textunderscore}box)}, see line 10 - 12). Then, the BB's position inside the baseline image of the provided WSI is determined (see line 13). Afterwards, the size of the BB is calculated for both dimension (see line 14). OpenSlide's \texttt{read{\textunderscore}region(location, level, size)} is used to access the ROI located at the specified position (the baseline image is at level 0\cite{DICOM10}, see line 15). The extracted image is then saved via the \texttt{save{\textunderscore}image(image, region, slide{\textunderscore}name)} function.

Once every region was extracted, the OpenSlide object is closed again.

\begin{lstlisting}[frame=single,language=python]
def wsi(file):
	slide = OpenSlide(file)
	slide_name = file.split('/')[-1]
	regions = read_json(file + '_' + DICTIONARY)
	
	for region in regions:
		if(TESSELLATE):
			tessellate_wsi(slide, slide_name, region)
		else:
			bounding_box = get_bounding_box(region)
			if(RESIZE):
				bounding_box = resize_bounding_box(bounding_box)
			location = (bounding_box['x_min'], bounding_box['y_min'])
			size = (bounding_box['x_max'] - bounding_box['x_min'], bounding_box['y_max'] - bounding_box['y_min'])
			image = slide.read_region(location, 0, size)
			save_image(image, region, slide_name)
	
	slide.close()
\end{lstlisting}


\subsubsection{tessellate{\textunderscore}wsi(slide, slide{\textunderscore}name, region)}
\texttt{tessellate{\textunderscore}wsi(slide, slide{\textunderscore}name, region)} realizes the approximation of a ROI through tessellation\footnote{
	Compare subsection \ref{sec5_tessellation}
}. To do so, it first gets the slide dimension to calculate the amount of virtual tiles necessary to cover the whole WSI (see line 2 - 4). Afterwards, a contour is created by iteration over every segment of a region's path and associating it with a virtual tile (see line 10 - 18). The result is a contour of virtual tiles.

In the next step, the RI is created. Each pixel in the RI corresponds to a virtual tile. Because of this relation between tiles and pixels, the created tile contour can be used to fill in the ROI in the RI via OpenCV's \texttt{cv2.drawContours(...)} function (see line 23). The provided parameters are:
\begin{itemize}
	\item \textbf{cv{\textunderscore}ref{\textunderscore}img}: the RI
	\item \textbf{[contour]}: the list of contours to draw
	\item \textbf{0}: which contour to draw from the provided array (0 equals all)
	\item \textbf{(255, 255, 255)}: a tupel with the color in which the contour is drawn in (white)
	\item \textbf{-1}: the stroke width of the contour (-1 equals fill)
\end{itemize}

The result is the RI where every pixel corresponding to a virtual tile in the ROI is white. In the next step, the function iterates over every single pixel of the RI and extracts a ROI into an individual image, whenever the corresponding pixel is white (see line 27 - 37).

After all tiles were extracted, an associated metadata file is created (see line 40).

\begin{lstlisting}[frame=single,language=python]
def tessellate_wsi(slide, slide_name, region):
	n,m = slide.dimensions
	m = m / TESSELLATE[HEIGHT]
	n = n / TESSELLATE[WIDTH]
	
	if SHOW:
		ox = 999999
		oy = 999999
	
	contour = []
	for coords in region.get('imgCoords'):
	if SHOW:
		if(coords.get('y') < oy): oy = coords.get('y')
		if(coords.get('x') < ox): ox = coords.get('x')
	x = int(coords.get('x') / TESSELLATE[WIDTH])
	y = int(coords.get('y') / TESSELLATE[HEIGHT])
	if [x, y] not in contour:
		contour.append([x, y])
	
	contour = np.asarray(contour)
	ref_img = Image.new('RGB', (n,m))
	cv_ref_img = np.array(ref_img)
	cv2.drawContours(cv_ref_img, [contour], 0, (255,255,255), -1)
	if SHOW:
		dbg_img = Image.new('RGB', (n,m))
	tiles = []
	for i in xrange(0, m):
		for j in xrange(0, n):
			px = cv_ref_img[i,j]
			if (px == [255, 255, 255]).all():
				location = ((j) * TESSELLATE[WIDTH], (i) * TESSELLATE[HEIGHT])
				size = TESSELLATE
				tile = slide.read_region(location, 0, size)
				tile_name = save_image(tile, region, slide_name, i, j)
				tiles.append(tile_name.split('/')[-1] + '.jpeg')
				if SHOW:
					dbg_img.paste(tile, (j * TESSELLATE[WIDTH] - int(ox), i * TESSELLATE[HEIGHT] - int(oy)))
	if SHOW:
	dbg_img.show()
	save_metadata(generate_file_name(region, slide_name), region, tiles)
\end{lstlisting}


\subsubsection{is{\textunderscore}supported(file)}
\texttt{is{\textunderscore}supported(file)} checks if a provided file is a proprietary WSI (see subsection \ref{sec2_proprietaryForams}).
\begin{lstlisting}[frame=single,language=python]
def is_suppoted(file):
	ext = (file.split('.'))[-1]
	if(
		'bif' in ext or
		'mrxs' in ext or
		'npdi' in ext or
		'scn' in ext or
		'svs' in ext or
		'svslide' in ext or
		'tif' in ext or
		'tiff' in ext or
		'vms' in ext or
		'vmu' in ext
	):
		return 1
	else:
		return 0
\end{lstlisting}


\section{DZI}

\subsubsection{dzi(file)}
\texttt{dzi(file)} extracts the regions from a DZI. Since OpenSlide can not be used with DZIs, the corresponding ROI must be stitched manually. To do so, the provided file's DZI metadata must be parsed to access the information about width, height, format and location of the highest resolution layer (see line 2 - 6). The saved regions are parsed with the \texttt{read{\textunderscore}json(path)} function (see line 7).

For each parsed region, an image is extracted together with a metadata file, like in the WSI scenario. If the tessellation parameter was provided when TessellationService.py was called, \texttt{tessellate{\textunderscore}dzi(dzi, slide{\textunderscore}name, region)} is called. Otherwise, the BB is calculated via \texttt{get{\textunderscore}bounding{\textunderscore}box(region)}. The tiles containing the ROI are stitched together and cropped to size in the \texttt{create{\textunderscore}image{\textunderscore}from{\textunderscore}tiles(dzi, bounding{\textunderscore}box)} function. The resulting image is saved via the \texttt{save{\textunderscore}image(image, region, slide{\textunderscore}name)} function (see line 10 - 15).

\begin{lstlisting}[frame=single,language=python]
def dzi(file):
	slide_name = file.split('/')[-1]
	with open(file, 'r') as dzi_file:
		content = dzi_file.read()
	root = ET.fromstring(content)
	dzi = {'tile_size': int(root.get('TileSize')), 'width': int(root[0].get('Width')), 'height': int(root[0].get('Height')), 'tile_source': get_tile_source(file), 'format': root.get('Format')}
	regions = read_json(file + '_' + DICTIONARY)
	
	for region in regions:
		if TESSELLATE:
			tessellate_dzi(dzi, slide_name, region)
		else:
		bounding_box = get_bounding_box(region)
		image = create_image_from_tiles(dzi, bounding_box)
		save_image(image, region, slide_name)
\end{lstlisting}


\subsubsection{create{\textunderscore}image{\textunderscore}from{\textunderscore}tiles(dzi, bounding{\textunderscore}box)}
\texttt{create{\textunderscore}image{\textunderscore}from{\textunderscore}tiles(dzi, bounding{\textunderscore}box)} stitches the tiles containing the ROI to extract together by utilizing the ROI's BB.

If the -r parameter was provided at the TS' execution, the BB will be adjusted to fit the provided image ratio. This also has influence on what tiles will be needed (see line 2 - 3). The stitching of the baseline image area is realized in \texttt{get{\textunderscore}tiles{\textunderscore}from{\textunderscore}bounding{\textunderscore}box(dzi, bounding{\textunderscore}box)} (see line 4). Once the tiles are stitched, the offset of the ROI's BB inside the stitched image is calculated (see line 6 - 7), as well as its corners (see line 9 - 12).

This information is used to crop the stitched image to the size of the ROI's BB.

\begin{lstlisting}[frame=single,language=python]
def create_image_from_tiles(dzi, bounding_box):
	if(RESIZE):
		bounding_box = resize_bounding_box(bounding_box)
	tile_image = get_tiles_from_bounding_box(dzi, bounding_box)
	
	offset_x = bounding_box['x_min']
	offset_y = bounding_box['y_min']
	
	x_min = bounding_box['x_min'] - offset_x
	x_max = bounding_box['x_max'] - offset_x
	y_min = bounding_box['y_min'] - offset_y
	y_max = bounding_box['y_max'] - offset_y
	
	return tile_image.crop((x_min, y_min, x_max, y_max))
\end{lstlisting}


\subsubsection{get{\textunderscore}tiles{\textunderscore}from{\textunderscore}bounding{\textunderscore}box(dzi, bounding{\textunderscore}box)}

\texttt{get{\textunderscore}tiles{\textunderscore}from{\textunderscore}bounding{\textunderscore}box(dzi, bounding{\textunderscore}box)} stitches together a baseline image area  based on the provided BB.

To do so, the corners of the BB are converted to tile positions (see line 2 - 5). Then, a new image is created which will hold the needed tiles (see line 7). All needed tiles are iterated in width and height and pasted into the corresponding postion of the holding image (see line 9 - 12).

The stitched image is then returned.

\begin{lstlisting}[frame=single,language=python]
def get_tiles_from_bounding_box(dzi, bounding_box):
	x_min = bounding_box['x_min'] / dzi['tile_size']
	x_max = bounding_box['x_max'] / dzi['tile_size']
	y_min = bounding_box['y_min'] / dzi['tile_size']
	y_max = bounding_box['y_max'] / dzi['tile_size']
	
	stitch = Image.new('RGB', ((x_max-x_min+1) * dzi['tile_size'], (y_max-y_min+1) * dzi['tile_size']))
	
	for i in range(x_min, x_max+1):
		for j in range(y_min, y_max+1):
			tile = Image.open(dzi['tile_source'] + str(i) + '_' + str(j) + '.' + dzi['format'])
			stitch.paste(tile, ((i - x_min) * dzi['tile_size'], (j - y_min) * dzi['tile_size']))
	return stitch
\end{lstlisting}


\subsubsection{get{\textunderscore}tile{\textunderscore}source(file)}
\texttt{get{\textunderscore}tile{\textunderscore}source(file)} finds the level with the highest resolution in a DZI's \emph{"{\textunderscore}files"} directory and builds a file path based on that.

\begin{lstlisting}[frame=single,language=python]
def get_tile_source(file):
	files_dir = file.replace('.dzi', '_files/')
	layers = os.listdir(files_dir)
	layers.remove('metadata.txt')
	layers = map(int, layers)
	return files_dir + str(max(layers)) + '/'
\end{lstlisting}


\subsubsection{tessellate{\textunderscore}dzi(dzi, slide{\textunderscore}name, region)}
\texttt{tessellate{\textunderscore}dzi(dzi, slide{\textunderscore}name, region)} works in the same way as \texttt{tessellate{\textunderscore}wsi(slide, slide{\textunderscore}name, region)}, except for the tile extraction (see line 26 - 35).

As in \texttt{dzi(file)}, a baseline image area must be stitched (see line 3). To extract a virtual tile from that stitched image, \texttt{tessellate{\textunderscore}dzi(dzi, slide{\textunderscore}name, region)} crops that stitched image to the size of the current virtual tile and saves this in an individual image.

Once all tiles are extracted, a metadata file is created (see line 43).

\begin{lstlisting}[frame=single,language=python]
def tessellate_dzi(dzi, slide_name, region):
	bounding_box = get_bounding_box(region)
	tile_image = get_tiles_from_bounding_box(dzi, bounding_box)
	
	offset_x = bounding_box['x_min']
	offset_y = bounding_box['y_min']
	
	n,m = tile_image.size
	
	m = m / TESSELLATE[HEIGHT]
	n = n / TESSELLATE[WIDTH]
	
	contour = []
	for coords in region.get('imgCoords'):
	x = int((coords.get('x') - offset_x) / TESSELLATE[WIDTH])
	y = int((coords.get('y') - offset_y) / TESSELLATE[HEIGHT])
	if [x, y] not in contour:
		contour.append([x, y])
	
	contour = np.asarray(contour)
	ref_img = Image.new('RGB', (n,m))
	cv_ref_img = np.array(ref_img)
	cv2.drawContours(cv_ref_img, [contour], 0, (255,255,255), -1)
	if SHOW:
		dbg_img = Image.new('RGB', tile_image.size)
	tiles = []
	for i in xrange(0, m):
		for j in xrange(0, n):
			px = cv_ref_img[i,j]
			if (px == [255, 255, 255]).all():
				tile = tile_image.crop((j * TESSELLATE[WIDTH] + (bounding_box['x_min'] % dzi['tile_size']),
				i * TESSELLATE[HEIGHT] + (bounding_box['y_min'] % dzi['tile_size']),
				j * TESSELLATE[WIDTH] + (bounding_box['x_min'] % dzi['tile_size']) + TESSELLATE[WIDTH],
				i * TESSELLATE[HEIGHT] + (bounding_box['y_min'] % dzi['tile_size']) + TESSELLATE[HEIGHT]))
				tile_name = save_image(tile, region, slide_name, i, j)
				tiles.append(tile_name.split('/')[-1] + '.jpeg')
				if SHOW:
					dbg_img.paste(tile, (j * TESSELLATE[WIDTH], i * TESSELLATE[HEIGHT]))
	if SHOW:
		dbg_img.show()
	save_metadata(generate_file_name(region, slide_name), region, tiles)
\end{lstlisting}


\section{Utility}

\subsubsection{read{\textunderscore}json(path)}

\begin{lstlisting}[frame=single,language=python]
def read_json(path):
	try:
		with open(path, 'r') as file:
		str = (file.read())
		data = json.loads(str.decode('utf-8'))
		return data
	except IOError:
		print('Could not load saved annotations from ' + path)
\end{lstlisting}


\subsubsection{save{\textunderscore}metadata(name, region, *tiles)}

\begin{lstlisting}[frame=single,language=python]
def save_metadata(name, region, *tiles):
	if len(tiles) > 0:
	name = name + '_tessellated.metadata.json'
	if not FORCE:
		cnt = 0
		while os.path.isfile(name):
			cnt+=1
		name = name + '(' + str(cnt) +')'
	else:
		image_name = name
	name = name + '.metadata.json'
	with open(name, 'w+') as file:
		data = {'label': region.get('name'), 'zoom': region.get('zoom'), 'context': region.get('context')}
	if len(tiles) > 0:
		data['tiles'] = tiles
	else:
		data['image'] = image_name.split('/')[-1] + '.jpeg'
	content = json.dumps(data, ensure_ascii=False)
	file.write(content.encode('utf-8'))
\end{lstlisting}


\subsubsection{generate{\textunderscore}file{\textunderscore}name(region, slide{\textunderscore}name, *tiles)}

\begin{lstlisting}[frame=single,language=python]
def generate_file_name(region, slide_name, *tiles):
	if(OUTPUT):
		dest = OUTPUT + region['name']
	else:
		dest = region['name']
	if not os.path.exists(dest):
		os.makedirs(dest)
	name = dest + '/' + slide_name + '_' + str(region['uid'])
	if len(tiles) > 0:
		for entry in tiles:
			name += "_" + str(entry)
	if not FORCE:
		cnt = 0
		while os.path.isfile(name):
			cnt+=1
		name = name + '(' + str(cnt) +')'
	return name
\end{lstlisting}


\subsubsection{save{\textunderscore}image(image, region, slide{\textunderscore}name, *tiles}

\begin{lstlisting}[frame=single,language=python]
def save_image(image, region, slide_name, *tiles):
	if len(tiles) == 0:
		name = generate_file_name(region, slide_name)
	else:
		name = generate_file_name(region, slide_name, tiles)
	if RESIZE:
		image = image.resize(RESIZE, INTERPOLATION)
	# L = R * 299/1000 + G * 587/1000 + B * 114/1000
	if GRAYSCALE:
		image = image.convert('L')
	image.save(name + '.jpeg', 'jpeg')
	if len(tiles) == 0:
		save_metadata(name, region)
	return name
\end{lstlisting}


\subsubsection{get{\textunderscore}bounding{\textunderscore}box(region)}

\begin{lstlisting}[frame=single,language=python]
def get_bounding_box(region):
	x_min = sys.float_info.max
	x_max = sys.float_info.min
	y_min = x_min
	y_max = x_max
	for coordinate in region.get('imgCoords'):
		x = coordinate.get('x')
		y = coordinate.get('y')
		if(x >= x_max):
			x_max = x
		if(x < x_min) :
			x_min = x
		if(y >= y_max):
			y_max = y
		if(y < y_min) :
			y_min = y
	
	return {'x_max': int(np.ceil(x_max)), 'x_min': int(np.floor(x_min)),
		'y_max': int(np.ceil(y_max)), 'y_min': int(np.floor(y_min))}
\end{lstlisting}


\subsubsection{resize{\textunderscore}bounding{\textunderscore}box(region)}

\begin{lstlisting}[frame=single,language=python]
def resize_bounding_box(bounding_box):
	r_ratio = RESIZE[WIDTH] / float(RESIZE[HEIGHT])
	bb_width = float(bounding_box['x_max'] - bounding_box['x_min'])
	bb_height = float(bounding_box['y_max'] - bounding_box['y_min'])
	bb_ratio = bb_width / bb_height
	if r_ratio == bb_ratio:
		return bounding_box
	else:
		if r_ratio == 1:
			# target is square
			s1 = bb_height/bb_width
			s2 = bb_width/bb_height
			scaled = min(bb_width, bb_height) * max(s1, s2) - min(bb_width, bb_height)
			if(bb_width > bb_height):
				bounding_box['y_min'] -= int(np.floor(scaled/2))
				bounding_box['y_max'] += int(np.ceil(scaled/2))
			else:
				bounding_box['x_min'] -= int(np.floor(scaled/2))
				bounding_box['x_max'] += int(np.ceil(scaled/2))
		elif r_ratio < 1:
			# target is higher than wide
			h_s = 1 / r_ratio
			if bb_height > (bb_width * h_s):
				# adjust width:
				w_new = (bb_height / h_s) - bb_width
				bounding_box['x_min'] -= int(np.floor(w_new/2))
				bounding_box['x_max'] += int(np.ceil(w_new/2))
			else:
				# adjust height:
				h_new = h_s * bb_width - bb_height
				bounding_box['y_min'] -= int(np.floor(h_new/2))
				bounding_box['y_max'] += int(np.ceil(h_new/2))
		else:
			# target is wider than high
			w_s = r_ratio
			if bb_width > (bb_height * w_s):
				# adjust height
				h_new = (bb_width / w_s) - bb_height
				bounding_box['y_min'] -= int(np.floor(h_new/2))
				bounding_box['y_max'] += int(np.ceil(h_new/2))
			else:
				# adjust width:
				w_new = w_s * bb_height - bb_width
				bounding_box['x_min'] -= int(np.floor(w_new/2))
				bounding_box['x_max'] += int(np.ceil(w_new/2))
		
		# check if bb is big enough
		bb_width = float(bounding_box['x_max'] - bounding_box['x_min'])
		if bb_width < RESIZE[WIDTH]:
			s = RESIZE[WIDTH] / bb_width
			bounding_box = scale_bounding_box(bounding_box, s)
		bb_height = float(bounding_box['y_max'] - bounding_box['y_min'])
		if bb_height < RESIZE[HEIGHT]:
			s = RESIZE[HEIGHT] / bb_height
			bounding_box = scale_bounding_box(bounding_box, s)
		
		if(bounding_box['y_min'] < 0):
			dif = bounding_box['y_min'] * (-1)
			bounding_box['y_min'] += dif
			bounding_box['y_max'] += dif
		if(bounding_box['x_min'] < 0):
			dif = bounding_box['x_min'] * (-1)
			bounding_box['x_min'] += dif
			bounding_box['x_max'] += dif
	
		return bounding_box
\end{lstlisting}


\subsubsection{scale{\textunderscore}bounding{\textunderscore}box(bounding{\textunderscore}, scale)}

\begin{lstlisting}[frame=single,language=python]
def scale_bounding_box(bounding_box, scale):
	bb_width = float(bounding_box['x_max'] - bounding_box['x_min'])
	add_w = (bb_width * scale) - bb_width
	bounding_box['x_min'] -= int(np.floor(add_w/2))
	bounding_box['x_max'] += int(np.ceil(add_w/2))
	
	bb_height = float(bounding_box['y_max'] - bounding_box['y_min'])
	add_h = (bb_height * scale) - bb_height
	bounding_box['y_min'] -= int(np.floor(add_h/2))
	bounding_box['y_max'] += int(np.ceil(add_h/2))
	
	return bounding_box
\end{lstlisting}
\end{appendices}

\bibliographystyle{plain}

\bibliography{./bibl/bibl}
\listoffigures
\listoftables
%TODO deine Nomenklatur hat grade sowohl Neural Network als auch Neural Networks als NN ausgezeichnet (siehe letzte seite des pdfs).
\printnomenclature
\chapter*{Statutory declaration}\markboth{Statutory declaration}{}
  \addcontentsline{toc}{chapter}{Statutory declaration}

I declare that I have developed and written the enclosed Master Thesis completely by myself, and have not used sources or means without declaration in the text. Any thoughts from others or literal quotations are clearly marked. The Master Thesis was not used in the same or in a similar version to achieve an academic grading or is being published elsewhere.
\\
\vspace{2cm}
\\
Hiermit versichere ich, dass ich die vorliegende Masterarbeit selbstständig und nur unter Verwendung der angegebenen Quellen und Hilfsmittel verfasst habe. Die Arbeit wurde bisher in gleicher oder ähnlicher Form keiner anderen Prüfungsbehörde vorgelegt.

\vspace{5cm}
\begin{tabbing}
\hspace{6cm}  \= \kill
\textbf{\textit{Berlin, den 23.09.2016} \hspace{5cm} \textit{Sascha Nawrot}}
\end{tabbing}



\end{document}
